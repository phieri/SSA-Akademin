\documentclass[a4paper,twoside,openright]{article}
%\documentclass[a4paper,oneside,openright]{book}
% Prepare for svenska tecken
\usepackage[T1]{fontenc}
\usepackage[swedish]{babel}
\usepackage{icomma}
\usepackage{lmodern}
\usepackage[numbered]{bookmark}
%\usepackage[]{geometry}
\addto\captionsswedish{\renewcommand{\figurename}{Bild}}
\usepackage{amsmath}
%\usepackage{fancyhdr}
%\usepackage{wrapfig}
\usepackage{caption}
\usepackage{framed}
\usepackage[fulladjust]{marginnote}
\usepackage{color}
\usepackage{amssymb}
\newcommand{\hilight}[1]{\colorbox{yellow}{#1}}
\usepackage[toctitles,raggedright]{titlesec}
\usepackage{hyperref}
\hypersetup{
    pdftitle={KonCEPT för amatörradiocertifikat},
    pdfauthor={Föreningen Sveriges Sändareamatörer},
    pdfkeywords={Sveriges Sändareamatörer, SSA, amatörradio},
    pdflang={sv},
    colorlinks,
    citecolor=black,
    filecolor=black,
    linkcolor=black,
    urlcolor=black
}
\usepackage{stmaryrd} % För symbolen \boxbox, kräver paketet texlive-math-extra
\usepackage{gensymb}
\usepackage{siunitx}
\sisetup{
    output-decimal-marker = {,},
    output-product = \cdot,
    retain-explicit-plus = true,
%    input-product = {*},
    product-units = single,
    per-mode = symbol,
    range-units = single,
    range-phrase = {--},
    separate-uncertainty=true,
    multi-part-units=single
}
\DeclareSIUnit\noop{\relax}
% % % % % % % % % % % 
% Detta är nya environments för review. De bör vara relativt självförklarande hur de används.
% I princip sätter man bara den del av texten som har en viss status mellan\begin{rev-granskat} och \end{rev-granskat} tex.
% Undvik att nästla dem för det är ingen idé det fungerar inte.
% De är testade med ett antal andra environemnt som tabular mm men kolla att det fungerar med de environments du använder.
% % % % % % % % % % % % % % % % % % % % % % % % % % % % % % % % % % % % % % % % % % % % % % % % % % % % % % % % % % % % % % % 
\usepackage[svgnames,rgb]{xcolor}
\usepackage{pdfcomment}
\newenvironment{rev-ogranskat}{\begin{pdfsidelinecomment}[color=black,linewidth=3px,caption=inline]{Ogranskat}}{\end{pdfsidelinecomment}}
\newenvironment{rev-omarbetas}{\begin{pdfsidelinecomment}[color=red,linewidth=3px,caption=inline]{Omarbetas}}{\end{pdfsidelinecomment}}
\newenvironment{rev-raderas}{\begin{pdfsidelinecomment}[color=red,linewidth=3px,caption=inline]{Raderas}}{\end{pdfsidelinecomment}}
\newenvironment{rev-redo}{\begin{pdfsidelinecomment}[color=yellow,linewidth=3px,caption=inline]{Redo att granska}}{\end{pdfsidelinecomment}}
\newenvironment{rev-granskat}[1][]%
{\begin{pdfsidelinecomment}[color=green,linewidth=3px,caption=inline]%
{Granskat #1}}%
{\end{pdfsidelinecomment}}
\newenvironment{rev-nytt}[1][]%
{\begin{pdfsidelinecomment}[color=brown,linewidth=3px,caption=inline]%
{Nytt #1}}%
{\end{pdfsidelinecomment}}
\newenvironment{rev-releasat}{\begin{pdfsidelinecomment}[color=blue,linewidth=3px,caption=inline]{Klart}}{\end{pdfsidelinecomment}}

%% k7per: Swap these two definitions if you want a small Orange blob
%% behind the ref locations for ease of layout work.
\newcommand\ssaref[1]{\ref{#1}}
%% \newcommand\ssaref[1]{\setlength{\fboxsep}{0pt}\colorbox{Orange}{\ref{#1}}}

\clubpenalty=9990
\widowpenalty=9999
\brokenpenalty=4999

\usepackage[europeanvoltages,europeancurrents,europeanresistors,cuteinductors,smartlabels]{circuitikz}
\usepackage[framemethod=TikZ]{mdframed}

\mdfdefinestyle{FactBox}{%
    linecolor=blue,
    outerlinewidth=2pt,
    roundcorner=20pt,
    innertopmargin=\baselineskip,
    innerbottommargin=\baselineskip,
    innerrightmargin=20pt,
    innerleftmargin=20pt,
    backgroundcolor=gray!50!white}
\newcommand{\infobox}[1]{
\begin{figure}%[r][0.5\textwidth]
  \begin{mdframed}[style=FactBox]
#1
  \end{mdframed}
\end{figure}
}

% Prepare for tables
\usepackage{multirow}
\usepackage{xtab}

% Prepare for lists
\usepackage{enumitem}

% Prepare for graphics
\usepackage{graphicx}

\raggedbottom

% Prepare for version handling
\usepackage{xstring}
\usepackage{catchfile}
\CatchFileDef{\HEAD}{SHA.txt}{}
\newcommand{\gitrevision}{%
  \StrLeft{\HEAD}{7}%
}
\CatchFileDef{\VERSION}{VERSION.txt}{}
\newcommand{\revision}{%
  \VERSION (\gitrevision)%
}

%% Frontpage background
\usepackage{eso-pic}
\newcommand\BackgroundPic{%
\put(0,0){%
\parbox[b][\paperheight]{\paperwidth}{%
\vfill
\centering
\includegraphics[width=\paperwidth,height=\paperheight,%
keepaspectratio]{images/koncept-front.jpg}%
\vfill
}}}


\newcommand\Backgroundtwo{%
\put(0,0){%
\parbox[b][\paperheight]{\paperwidth}{%
\vfill
\centering
\includegraphics[width=\paperwidth,height=\paperheight,%
keepaspectratio]{images/koncept-larobok-front.jpg}%
\vfill
}}}



\newcommand\BackgroundPicLast{%
\put(0,0){%
\parbox[b][\paperheight]{\paperwidth}{%
\centering%
\vfill%
%\includegraphics[width=6cm]{images/isbn_9789186368234}%
%\includegraphics[width=\paperwidth,height=\paperheight]{images/koncept-back.pdf}%
\vfill%
}}}

%tlfällig fix för kompilering%
\nonstopmode
\usepackage[swedish]{babel}

\headheight = 12.6pt

% Prepare for abstract
\pagestyle{plain}
\newenvironment{abstract}%
{\cleardoubblanklepage\null \vfill\begin{center}\bfseries Abstract \end{center}}%
%{\cleardoublepage\null \vfill\begin{center}%
%\bfseries \abstractname \end{center}}%
     {\vfill\null}

\usepackage{refbook}

% Prepare for index
\usepackage{makeidx}
\makeindex

\begin{document}
\frontmatter
%\mainmatter
% Frontpage

\title{Kursplan för Koncept}
\author{Magnus Danielson SA0MAD}
\maketitle

\section*{Introduktion}

Denna kursplan baseras på den kurs som Täby Sändaramatörer TSA/SK0MT håller.
Den är uppdelad i 14 lektioner.
Ellära och reglemente lektionerna ligger om varandra för att eleverna skall
hinna smälta saker och läsa ikapp.
Ett praktiskt moment är inlagt för att bygga vanor och släppa mikronfon-rädslan.
På slutet finns ett tillfälle för repetition och frågor, samt ett sista
tillfälle där ett prov gås igenom.

Läsanvisningar till boken ``Koncept för Amatörradiocertifikat, andra upplagan''
anges per lektion, jämte de däri liggande HAREC-kraven.
Som appendix finns även de detaljerade HAREC-kraven, så man som referens kan se
i vilken lektion som de hanteras.
Detta försäkrar att alla kraven kan täckas av lektioner.

\lfoot[\revision]{}
\rfoot[]{\revision}

\cleardoublepage
\pagestyle{fancy}


\tableofcontents

\setlength{\parindent}{0pt}
\setlength{\parskip}{1ex plus 0.5ex minus 0.2ex}

\mainmatter

\section{Introduktion och Ellära}

\subsection{Introduktion}

Kort introduktion om kursen, kursledare och lärare samt det praktiska.
Utdelning av böcker. Sen följer en mjukstart för att få med sig de flesta.

\subsection{Matematik}
Bilaga B.1 Uttryck
\textbf{HAREC I.\ref{HAREC.I.c.1}\label{myHAREC.I.c.1}}
\textbf{HAREC I.\ref{HAREC.I.c.2}\label{myHAREC.I.c.2}}
\textbf{HAREC I.\ref{HAREC.I.c.6}\label{myHAREC.I.c.6}}

Bilaga B.2 Formler
\textbf{HAREC I.\ref{HAREC.I.d}\label{myHAREC.I.d.2a}}

Bilaga B.3 Exempel med 1 obekant
\textbf{HAREC I.\ref{HAREC.I.d}\label{myHAREC.I.d.2b}}

Bilaga B.5 Potenser, digniteter
\textbf{HAREC I.\ref{HAREC.I.c.4}\label{myHAREC.I.c.4}}

Bilaga B.6 Rötter
\textbf{HAREC I.\ref{HAREC.I.c.5}\label{myHAREC.I.c.5}}

Bilaga B.7 Logaritmer
\textbf{HAREC I.\ref{HAREC.I.c.3}\label{myHAREC.I.c.3}}

Bilaga B.8 Binära tal
\textbf{HAREC I.\ref{HAREC.I.c.8}\label{myHAREC.I.c.8}}

\subsection{Ellära}

Kapitel 1.1.4 Konduktivitet - Ledare, halvledare och isolator
\textbf{HAREC a.\ref{HAREC.a.1.1}\label{myHAREC.a.1.1}, a.\ref{HAREC.a.1.1.1}\label{myHAREC.a.1.1.1}}

Kapitel 1.1.5 Elektrisk spänning -- Enheten volt
\textbf{HAREC a.\ref{HAREC.a.1.1.2}\label{myHAREC.a.1.1.2b}, a.\ref{HAREC.a.1.1.3}\label{myHAREC.a.1.1.3b}}

Kapitel 1.1.7 Elektrisk ström -- Enheten ampere
\textbf{HAREC a.\ref{HAREC.a.1.1.2}\label{myHAREC.a.1.1.2a}, a.\ref{HAREC.a.1.1.3}\label{myHAREC.a.1.1.3a}}

Kapitel 1.1.10 Resistans -- Enheten ohm
\textbf{HAREC a.\ref{HAREC.a.1.1.2}\label{myHAREC.a.1.1.2c}, a.\ref{HAREC.a.1.1.3}\label{myHAREC.a.1.1.3c}}

Kapitel 1.1.11 Ohms lag
\textbf{HAREC a.\ref{HAREC.a.1.1.4}\label{myHAREC.a.1.1.4}}

Kapitel 1.1.12 Kirchhoffs lagar
\textbf{HAREC a.\ref{HAREC.a.1.1.5}\label{myHAREC.a.1.1.5}}

Kapitel 1.1.13 Elektrisk effekt -- Enheten watt
\textbf{HAREC a.\ref{HAREC.a.1.1.6}\label{myHAREC.a.1.1.6}, a.\ref{HAREC.a.1.1.7}\label{myHAREC.a.1.1.7}}

Kapitel 1.1.15 Joules lag
\textbf{HAREC a.\ref{HAREC.a.1.1.8}\label{myHAREC.a.1.1.8}}

Kapitel 1.1.17 Amperetimmar (Ah) och batterikapacitet
\textbf{HAREC a.\ref{HAREC.a.1.1.9}\label{myHAREC.a.1.1.9}}

Kapitel 1.2.1 Elektromotorisk kraft -- EMK
\textbf{HAREC a.\ref{HAREC.a.1.2}\label{myHAREC.a.1.2}, a.\ref{HAREC.a.1.2.1}\label{myHAREC.a.1.2.1}} 

Kapitel 1.2.2 Serie- och parallellkopplade kraftkällor
\textbf{HAREC a.\ref{HAREC.a.1.2.2}\label{myHAREC.a.1.2.2}}

\section{Amatörradiotermer & Reglemente}

\section{Ellära - DC}
\section{Ellära - AC}

\section{Elsäkerhet}
\section{Ellära - radioteknik}
\section{Radiosändare}
\section{Radiomottagare}
\section{Praktik HF/VHF}
\section{Antenner}

\section{Vågutbredning}
\section{EMC/EMF}
\section{Repetition, frågor}
\section{Genomgång av övningsskrivning}

\appendix

\chapter{Kunskapskrav i CEPT HAREC}
\label{CEPT HAREC}

%% k7per
\noindent
Den här upplagan av KonCEPT är baserad på CEPT T/R 61-02 Harmonised Amateur
Radio Examination Certificate (HAREC) Edition 4 June 2016 \cite{TR6102}.

Själva kunskapskraven i HAREC finns beskrivna i
''Annex 6: Examination syllabus and requirements for a HAREC''.
Alla krav i den finns med här, i HAREC originalformulering, enbart omarbetad
med avseende på format.
För varje krav redovisas sedan en eller flera referenser i den övriga texten
där det kravet anses uppfyllas.

I bokens text står det \textbf{HAREC a.\ssaref{HAREC.a.1.4}} där magnetfält
behandlas för att referera till HAREC-kravet 1.4 Magnetic field;.
Samma ställe refereras sedan från kravet med delkapitel inom parentes
(\ssaref{myHAREC.a.1.4}).
I förekommande fall kan ett krav vara uppdelat till flera referenser för att
det täcks på flera separata ställen i texten.

I PDF-utgåvan är dessa referenser länkar som man kan klicka på för smidig
navigering.

\section{Introduction}

\makeatletter
\renewcommand{\theenumii}{\arabic{enumii}}
\renewcommand{\labelenumii}{\theenumi.\theenumii}
\renewcommand{\p@enumii}{\theenumi.}

\renewcommand{\theenumiii}{\arabic{enumiii}}
\renewcommand{\labelenumiii}{\theenumi.\theenumii.\theenumiii}
\renewcommand{\p@enumiii}{\theenumi.\theenumii.}
\makeatother

\begin{enumerate}[label=\alph*]
\item Where quantities are referred to, candidates should know the units in
  which these quantities are expressed, as well as the generally used multiples
  and sub-multiples of these units.
\item Candidates must be familiar with the compound of the symbols.
\item Candidates must know the following mathematical concepts and operations:
\begin{enumerate}
\item adding, subtracting, multiplying and dividing (\ssaref{myHAREC.I.c.1})\label{HAREC.I.c.1}
\item fractions (\ssaref{myHAREC.I.c.2})\label{HAREC.I.c.2}
\item powers of ten, exponentials, logarithms (\ssaref{myHAREC.I.c.3})\label{HAREC.I.c.3}
\item squaring (\ssaref{myHAREC.I.c.4})\label{HAREC.I.c.4}
\item square roots (\ssaref{myHAREC.I.c.5})\label{HAREC.I.c.5}
\item inverse values (\ssaref{myHAREC.I.c.6})\label{HAREC.I.c.6}
\item interpretation of linear and non-linear graphs
\item binary number system (\ssaref{myHAREC.I.c.8})\label{HAREC.I.c.8}
\end{enumerate}
\item Candidates must be familiar with the formulae used in this syllabus and
  be able to transpose them. (\ssaref{myHAREC.I.d.2a}, \ssaref{myHAREC.I.d.2b})\label{HAREC.I.d}
\end{enumerate}

\textbf{Not: Dessa övergripande krav finns spridda i boken och förväntas vara
uppfyllda.}

\section{Technical Content}

\makeatletter
\renewcommand{\labelenumi}{\theenumi.}

\renewcommand{\theenumii}{\arabic{enumii}}
\renewcommand{\labelenumii}{\theenumi.\theenumii}
\renewcommand{\p@enumii}{\theenumi.}

\renewcommand{\theenumiii}{\arabic{enumiii}}
\renewcommand{\labelenumiii}{\theenumi.\theenumii.\theenumiii}
\renewcommand{\p@enumiii}{\theenumi.\theenumii.}

\renewcommand{\theenumiv}{\arabic{enumiv}}
\renewcommand{\labelenumiv}{\theenumi.\theenumii.\theenumiii.\theenumiv}
\renewcommand{\p@enumiv}{\theenumi.\theenumii.\theenumiii.}
\makeatother

%% k7per, allow ragged right in all these references.
\begin{flushleft}
  
\begin{enumerate}
\item Electrical, electro-magnetic and radio theory
\begin{enumerate}

\item Conductivity; (\ssaref{myHAREC.a.1.1})\label{HAREC.a.1.1}
\begin{enumerate}
\item Conductor, semiconductor and insulator; (\ssaref{myHAREC.a.1.1.1})\label{HAREC.a.1.1.1}
\item Current (\ssaref{myHAREC.a.1.1.2a}), voltage (\ssaref{myHAREC.a.1.1.2b}) and resistance; (\ssaref{myHAREC.a.1.1.2c})\label{HAREC.a.1.1.2}
\item The units ampere (\ssaref{myHAREC.a.1.1.3a}), volt (\ssaref{myHAREC.a.1.1.3b}) and ohm; (\ssaref{myHAREC.a.1.1.3c})\label{HAREC.a.1.1.3}
\item Ohm's Law  \(\left[E = I \cdot R\right]\); (\ssaref{myHAREC.a.1.1.4}, \ssaref{myHAREC.a.1.1.4b})\label{HAREC.a.1.1.4}
\item Kirchhoff's Laws; (\ssaref{myHAREC.a.1.1.5})\label{HAREC.a.1.1.5}
\item Electric power \(\left[P = E \cdot I\right]\); (\ssaref{myHAREC.a.1.1.6})\label{HAREC.a.1.1.6}
\item The unit watt; (\ssaref{myHAREC.a.1.1.7})\label{HAREC.a.1.1.7}
\item Electric energy \(\left[W = P \cdot t\right]\); (\ssaref{myHAREC.a.1.1.8})\label{HAREC.a.1.1.8}
\item The capacity of a battery [ampere-hour]. (\ssaref{myHAREC.a.1.1.9})\label{HAREC.a.1.1.9}
\end{enumerate}

\item Sources of electricity; (\ssaref{myHAREC.a.1.2})\label{HAREC.a.1.2}
\begin{enumerate}
\item Voltage source, source voltage [EMF], short circuit current, internal resistance and terminal voltage; (\ssaref{myHAREC.a.1.2.1})\label{HAREC.a.1.2.1}
\item Series and parallel connection of voltage sources. (\ssaref{myHAREC.a.1.2.2})\label{HAREC.a.1.2.2}
\end{enumerate}

\item Electric field; (\ssaref{myHAREC.a.1.3})\label{HAREC.a.1.3}
\begin{enumerate}
\item Electric field strength; (\ssaref{myHAREC.a.1.3.1})\label{HAREC.a.1.3.1}
\item The unit volt/meter; (\ssaref{myHAREC.a.1.3.2})\label{HAREC.a.1.3.2}
\item Shielding of electric fields. (\ssaref{myHAREC.a.1.3.3})\label{HAREC.a.1.3.3}
\end{enumerate}

\item Magnetic field; (\ssaref{myHAREC.a.1.4})\label{HAREC.a.1.4}
\begin{enumerate}
\item Magnetic field surrounding live conductor; (\ssaref{myHAREC.a.1.4.1})\label{HAREC.a.1.4.1}
\item Shielding of magnetic fields. (\ssaref{myHAREC.a.1.4.2})\label{HAREC.a.1.4.2}
\end{enumerate}

\item Electromagnetic field; (\ssaref{myHAREC.a.1.5})\label{HAREC.a.1.5}
\begin{enumerate}
\item Radio waves as electromagnetic waves; (\ssaref{myHAREC.a.1.5.1})\label{HAREC.a.1.5.1}1
\item Propagation velocity and its relation with frequency and wavelength
  \(\left[v = \lambda \cdot f\right]\);
  (\ssaref{myHAREC.a.1.5.2})\label{HAREC.a.1.5.2}
\item Polarisation. (\ssaref{myHAREC.a.1.5.3})\label{HAREC.a.1.5.3}
\end{enumerate}

\item Sinusoidal signals; (\ssaref{myHAREC.a.1.6})\label{HAREC.a.1.6}
\begin{enumerate}
\item The graphic representation in time;
  (\ssaref{myHAREC.a.1.6.1})\label{HAREC.a.1.6.1}
\item Instantaneous value (\ssaref{myHAREC.a.1.6.2a})\label{HAREC.a.1.6.2a},
  amplitude \(\mathrm{[E_{max}]}\)
  (\ssaref{myHAREC.a.1.6.2b})\label{HAREC.a.1.6.2b},
  effective [RMS] value (\ssaref{myHAREC.a.1.6.2c})\label{HAREC.a.1.6.2c}
  and average value \(\left[U_{eff} = \dfrac{U_{max}}{\sqrt{2}}\right]\)
  (\ssaref{myHAREC.a.1.6.2d})\label{HAREC.a.1.6.2d};
\item Period (\ssaref{myHAREC.a.1.6.3a})\label{HAREC.a.1.6.3a}
  and duration of period (\ssaref{myHAREC.a.1.6.3b})\label{HAREC.a.1.6.3b};
\item Frequency; (\ssaref{myHAREC.a.1.6.4})\label{HAREC.a.1.6.4}
\item The unit hertz; (\ssaref{myHAREC.a.1.6.5})\label{HAREC.a.1.6.5}
\item Phase difference. (\ssaref{myHAREC.a.1.6.6})\label{HAREC.a.1.6.6}
\end{enumerate}

\item Non-sinusoidal signals, noise; (\ssaref{myHAREC.a.1.7})\label{HAREC.a.1.7}
\begin{enumerate}
\item Audio signals; (\ssaref{myHAREC.a.1.7.1})\label{HAREC.a.1.7.1}
\item Square wave; (\ssaref{myHAREC.a.1.7.2})\label{HAREC.a.1.7.2}
\item The graphic representation in time;
  (\ssaref{myHAREC.a.1.7.3})\label{HAREC.a.1.7.3}
\item D.C. voltage component (\ssaref{myHAREC.a.1.7.4a})\label{HAREC.a.1.7.4a},
  fundamental wave and higher harmonics
  (\ssaref{myHAREC.a.1.7.4b})\label{HAREC.a.1.7.4b};
\item Noise \(\left[P_N=TB\right]\) (receiver thermal noise, band noise,
  noise density, noise power in receiver bandwidth).
  (\ssaref{myHAREC.a.1.7.5})\label{HAREC.a.1.7.5}
\end{enumerate}

\item Modulated signals; (\ssaref{myHAREC.a.1.8})\label{HAREC.a.1.8}
\begin{enumerate}
\item CW; (\ssaref{myHAREC.a.1.8.1})\label{HAREC.a.1.8.1}
\item Amplitude modulation; (\ssaref{myHAREC.a.1.8.2})\label{HAREC.a.1.8.2}
\item Phase modulation (\ssaref{myHAREC.a.1.8.3a})\label{HAREC.a.1.8.3a},
  frequency modulation (\ssaref{myHAREC.a.1.8.3b})\label{HAREC.a.1.8.3b}
  and single-sideband modulation (\ssaref{myHAREC.a.1.8.3c})\label{HAREC.a.1.8.3c};
\item Frequency deviation and modulation index
  \(\left[m = \frac{\Delta F}{f_{mod}}\right]\);
  (\ssaref{myHAREC.a.1.8.4})\label{HAREC.a.1.8.4}
\item Carrier, sidebands and bandwidth;
  (\ssaref{myHAREC.a.1.8.5a}, \ssaref{myHAREC.a.1.8.5b})\label{HAREC.a.1.8.5}
\item Waveforms of CW (\ssaref{myHAREC.a.1.8.6a})\label{HAREC.a.1.8.6a},
  AM (\ssaref{myHAREC.a.1.8.6b})\label{HAREC.a.1.8.6b},
  SSB (\ssaref{myHAREC.a.1.8.6c})\label{HAREC.a.1.8.6c}
  and FM (\ssaref{myHAREC.a.1.8.6d})\label{HAREC.a.1.8.6d}
  signals (graphical presentation);
\item Spectrum of CW (\ssaref{myHAREC.a.1.8.7a})\label{HAREC.a.1.8.7a},
  AM (\ssaref{myHAREC.a.1.8.7b})\label{HAREC.a.1.8.7b}
  and SSB (\ssaref{myHAREC.a.1.8.7c})\label{HAREC.a.1.8.7c}
  signals (graphical presentation);
\item Digital modulations (\ssaref{myHAREC.a.1.8.8})\label{HAREC.a.1.8.8}:
  FSK (\ssaref{myHAREC.a.1.8.8a})\label{HAREC.a.1.8.8a},
  2-PSK (\ssaref{myHAREC.a.1.8.8b})\label{HAREC.a.1.8.8b},
  4-PSK (\ssaref{myHAREC.a.1.8.8c})\label{HAREC.a.1.8.8c},
  QAM (\ssaref{myHAREC.a.1.8.8d})\label{HAREC.a.1.8.8d};
\item Digital modulation (\ssaref{myHAREC.a.1.8.9})\label{HAREC.a.1.8.9}:
  bit rate (\ssaref{myHAREC.a.1.8.9a})\label{HAREC.a.1.8.9a},
  symbol rate (Baud rate) (\ssaref{myHAREC.a.1.8.9b})\label{HAREC.a.1.8.9b}
  and bandwidth (\ssaref{myHAREC.a.1.8.9c})\label{HAREC.a.1.8.9c};
\item CRC (\ssaref{myHAREC.a.1.8.10a})\label{HAREC.a.1.8.10a}
  and retransmissions (e.g. packet radio)
  (\ssaref{myHAREC.a.1.8.10b})\label{HAREC.a.1.8.10b},
  forward error correction (e.g. Amtor FEC)
  (\ssaref{myHAREC.a.1.8.10c})\label{HAREC.a.1.8.10c}.
\end{enumerate}

\item Power and energy; (\ssaref{myHAREC.a.1.9})\label{HAREC.a.1.9}
\begin{enumerate}
\item The power of sinusoidal signals\\
  \(\left[P=i^2 \cdot R; P=\frac{u^2}{R}; u=U_{eff}; i=I_{eff}\right]\);
  (\ssaref{myHAREC.a.1.9.1})\label{HAREC.a.1.9.1}
\item Power ratios corresponding to the following dB values:
  0~dB, 3~dB, 6~dB, 10~dB and 20~dB [both positive and negative];
  (\ssaref{myHAREC.a.1.9.2})\label{HAREC.a.1.9.2}
\item The input/output power ratio in dB of series-connected amplifiers and/or
  attenuators; (\ssaref{myHAREC.a.1.9.3})\label{HAREC.a.1.9.3}
\item Matching [maximum power transfer];
  (\ssaref{myHAREC.a.1.9.4})\label{HAREC.a.1.9.4}
\item The relation between power input and output and efficiency
  \(\left[\eta=\frac{P_{out}}{P_{in}}\cdot 100\%\right]\);
  (\ssaref{myHAREC.a.1.9.5})\label{HAREC.a.1.9.5}
\item Peak Envelope Power [\pep]. (\ssaref{myHAREC.a.1.9.6})\label{HAREC.a.1.9.6}
\end{enumerate}

\item Digital signal processing (DSP).
\begin{enumerate}
\item sampling and quantization; (\ssaref{myHAREC.a.1.10.1})\label{HAREC.a.1.10.1}
\item minimum sampling rate (Nyquist frequency);
  (\ssaref{myHAREC.a.1.10.2})\label{HAREC.a.1.10.2}
\item convolution (time domain / frequency domain, graphical presentation);
  (\ssaref{myHAREC.a.1.10.3})\label{HAREC.a.1.10.3}
\item anti-aliasing filtering, reconstruction filtering;
  (\ssaref{myHAREC.a.1.10.4})\label{HAREC.a.1.10.4}
\item ADC / DAC. (\ssaref{myHAREC.a.1.10.5})\label{HAREC.a.1.10.5}
\end{enumerate}

\end{enumerate}

\item Components
\begin{enumerate}

\item Resistor; (\ssaref{myHAREC.a.2.1})\label{HAREC.a.2.1}
\begin{enumerate}
\item The unit ohm; (\ssaref{myHAREC.a.2.1.1})\label{HAREC.a.2.1.1}
\item Resistance; (\ssaref{myHAREC.a.2.1.2})\label{HAREC.a.2.1.2}
\item Current/voltage characteristic;
  (\ssaref{myHAREC.a.2.1.3})\label{HAREC.a.2.1.3}
\item Power dissipation. (\ssaref{myHAREC.a.2.1.4})\label{HAREC.a.2.1.4}
\end{enumerate}

\item Capacitor; (\ssaref{myHAREC.a.2.2})\label{HAREC.a.2.2}
\begin{enumerate}
\item Capacitance; (\ssaref{myHAREC.a.2.2.1})\label{HAREC.a.2.2.1}
\item The unit farad; (\ssaref{myHAREC.a.2.2.2})\label{HAREC.a.2.2.2}
\item The relation between capacitance, dimensions and dielectric.
  (Qualitative treatment only); (\ssaref{myHAREC.a.2.2.3})\label{HAREC.a.2.2.3}
\item The reactance \(\left[X_C = \frac{1}{2\pi f \cdot C}\right]\);
  (\ssaref{myHAREC.a.2.2.4})\label{HAREC.a.2.2.4}
\item Phase relation between voltage and current.
  (\ssaref{myHAREC.a.2.2.5})\label{HAREC.a.2.2.5}
\end{enumerate}

\item Coil; (\ssaref{myHAREC.a.2.3})\label{HAREC.a.2.3}
\begin{enumerate}
\item Self-inductance; (\ssaref{myHAREC.a.2.3.1})\label{HAREC.a.2.3.1}
\item The unit henry; (\ssaref{myHAREC.a.2.3.2})\label{HAREC.a.2.3.2}
\item The effect of number of turns, diameter, length and core material on
  inductance. (Qualitative treatment only);
  (\ssaref{myHAREC.a.2.3.3})\label{HAREC.a.2.3.3}
\item The reactance  \(\left[X_L = 2\pi f \cdot L\right]\);
  (\ssaref{myHAREC.a.2.3.4})\label{HAREC.a.2.3.4}
\item Phase relation between current and voltage;
  (\ssaref{myHAREC.a.2.3.5})\label{HAREC.a.2.3.5}
\item Q-factor. (\ssaref{myHAREC.a.2.3.6})\label{HAREC.a.2.3.6}
\end{enumerate}

\item Transformers application and use; (\ssaref{myHAREC.a.2.4})\label{HAREC.a.2.4}
\begin{enumerate}
\item Ideal transformer \(\left[P_{prim} = P_{sec}\right]\);
  (\ssaref{myHAREC.a.2.4.1})\label{HAREC.a.2.4.1}
\item The relation between turn ratio and:
\begin{enumerate}
\item voltage ratio
  \(\left[\frac{u_{sec}}{u_{prim}} = \frac{n_{sec}}{n_{prim}}\right]\);
  (\ssaref{myHAREC.a.2.4.2.1})\label{HAREC.a.2.4.2.1}
\item current ratio
  \(\left[\frac{i_{sec}}{i_{prim}} = \frac{n_{prim}}{n_{sec}}\right]\);
  (\ssaref{myHAREC.a.2.4.2.2})\label{HAREC.a.2.4.2.2}
\item impedance ratio. (Qualitative treatment only);
  (\ssaref{myHAREC.a.2.4.2.3})\label{HAREC.a.2.4.2.3}
\item Transformers. (\ssaref{myHAREC.a.2.4.2.4})\label{HAREC.a.2.4.2.4}
\end{enumerate}
\end{enumerate}

\item Diode; (\ssaref{myHAREC.a.2.5})\label{HAREC.a.2.5}
\begin{enumerate}
\item Use and application of diodes:
\begin{enumerate}
\item Rectifier diode, zener diode, LED [light-emitting diode],
  voltage-variable and capacitor [varicap];
  (\ssaref{myHAREC.a.2.5.1.1})\label{HAREC.a.2.5.1.1}
\item Reverse voltage and leakage current.
  (\ssaref{myHAREC.a.2.5.1.2})\label{HAREC.a.2.5.1.2}
\end{enumerate}
\end{enumerate}

\item Transistor; (\ssaref{myHAREC.a.2.6})\label{HAREC.a.2.6}
\begin{enumerate}
\item PNP- (\ssaref{myHAREC.a.2.6.1a})\label{HAREC.a.2.6.1a}
  and NPN-transistor (\ssaref{myHAREC.a.2.6.1b})\label{HAREC.a.2.6.1b};
\item Amplification factor; (\ssaref{myHAREC.a.2.6.2})\label{HAREC.a.2.6.2}
\item Field effect vs. bipolar transistor (voltage vs. current driven);
  (\ssaref{myHAREC.a.2.6.3})\label{HAREC.a.2.6.3}
\item The transistor in the:
\begin{enumerate}
\item common emitter [source] circuit;
  (\ssaref{myHAREC.a.2.6.4.1})\label{HAREC.a.2.6.4.1}
\item common base [gate] circuit;
  (\ssaref{myHAREC.a.2.6.4.2})\label{HAREC.a.2.6.4.2}
\item common collector [drain] circuit;
  (\ssaref{myHAREC.a.2.6.4.3})\label{HAREC.a.2.6.4.3}
\item input and output impedances of the above circuits.
  (\ssaref{myHAREC.a.2.6.4.4})\label{HAREC.a.2.6.4.4}
\end{enumerate}
\end{enumerate}

\item Heat dissipation; (\ssaref{myHAREC.a.2.7})\label{HAREC.a.2.7}
\begin{enumerate}
\item heat conduction (\ssaref{myHAREC.a.2.7.1})\label{HAREC.a.2.7.1}
\item heat convection (\ssaref{myHAREC.a.2.7.2})\label{HAREC.a.2.7.2}
\item heat in transistors (\ssaref{myHAREC.a.2.7.3})\label{HAREC.a.2.7.3}
\item overheating and consequences (\ssaref{myHAREC.a.2.7.4})\label{HAREC.a.2.7.4}
\end{enumerate}

\item Miscellaneous.
\begin{enumerate}
\item Simple thermionic device [valve];
  (\ssaref{myHAREC.a.2.8.1})\label{HAREC.a.2.8.1}
\item Voltages and impedances in high power valve stages,
  impedance transformation; (\ssaref{myHAREC.a.2.8.2})\label{HAREC.a.2.8.2}
\item Simple integrated circuits (include opamps).
  (\ssaref{myHAREC.a.2.8.3})\label{HAREC.a.2.8.3}
\end{enumerate}
\end{enumerate}

\item Circuits
\begin{enumerate}

\item Combination of components;
\begin{enumerate}
\item Series and parallel circuits of resistors
  (\ssaref{myHAREC.a.3.1.1a})\label{HAREC.a.3.1.1a}
  (\ssaref{myHAREC.a.3.1.1b})\label{HAREC.a.3.1.1b},
  coils
  (\ssaref{myHAREC.a.3.1.1c})\label{HAREC.a.3.1.1c}
  (\ssaref{myHAREC.a.3.1.1d})\label{HAREC.a.3.1.1d},
  capacitors
  (\ssaref{myHAREC.a.3.1.1e})\label{HAREC.a.3.1.1e}
  (\ssaref{myHAREC.a.3.1.1f})\label{HAREC.a.3.1.1f},
  transformers and diodes
  (\ssaref{myHAREC.a.3.1.1g})\label{HAREC.a.3.1.1g}
  (\ssaref{myHAREC.a.3.1.1h})\label{HAREC.a.3.1.1h};
\item Current and voltage in these circuits;
  (\ssaref{myHAREC.a.3.1.2a}, \ssaref{myHAREC.a.3.1.2b})\label{HAREC.a.3.1.2c}
\item Behaviour of real (non-ideal) resistor, capacitor and inductors at
  high frequencies.
  (\ssaref{myHAREC.a.3.1.3a}, \ssaref{myHAREC.a.3.1.3b}, \ssaref{myHAREC.a.3.1.3c},
  \ssaref{myHAREC.a.3.1.3d})\label{HAREC.a.3.1.3e}
\end{enumerate}

\item Filter; (\ssaref{myHAREC.a.3.2})\label{HAREC.a.3.2}
\begin{enumerate}
\item Series-tuned and parallel-tuned circuit:
  (\ssaref{myHAREC.a.3.2.1}, \ssaref{myHAREC.a.3.2.1b})\label{HAREC.a.3.2.1}
\item Impedance;
  (\ssaref{myHAREC.a.3.2.2}, \ssaref{myHAREC.a.3.2.2b})\label{HAREC.a.3.2.2}
\item Frequency characteristic;  (\ssaref{myHAREC.a.3.2.3})\label{HAREC.a.3.2.3}
\item Resonance frequency \(\left[f=\frac{1}{2\pi\sqrt{LC}}\right]\);
  (\ssaref{myHAREC.a.3.2.4}, \ssaref{myHAREC.a.3.2.4b})\label{HAREC.a.3.2.4}
\item Quality factor of a tuned circuit
%% \\
  \(\left[Q=\dfrac{2\pi f \cdot L}{R_S};
%% \\
    Q=\dfrac{R_P}{2\pi f \cdot L}; Q=\dfrac{f_{res}}{B}\right]\);
  (\ssaref{myHAREC.a.3.2.5})\label{HAREC.a.3.2.5}
\item Bandwidth; (\ssaref{myHAREC.a.3.2.6})\label{HAREC.a.3.2.6}
\item Band-pass filter; (\ssaref{myHAREC.a.3.2.7})\label{HAREC.a.3.2.7}
\item Low-pass (\ssaref{myHAREC.a.3.2.8a})\label{HAREC.a.3.2.8a},
  high-pass (\ssaref{myHAREC.a.3.2.8b})\label{HAREC.a.3.2.8b},
  band-pass (\ssaref{myHAREC.a.3.2.8c})\label{HAREC.a.3.2.8c}
  and band-stop (\ssaref{myHAREC.a.3.2.8d})\label{HAREC.a.3.2.8d}
  filters composed of passive elements;
\item Frequency response; (\ssaref{myHAREC.a.3.2.9}, \ssaref{myHAREC.a.3.2.9a},
  \ssaref{myHAREC.a.3.2.9b}, \ssaref{myHAREC.a.3.2.9c}, \ssaref{myHAREC.a.3.2.9d},
  \ssaref{myHAREC.a.3.2.9e})\label{HAREC.a.3.2.9}
\item Pi filter (\ssaref{myHAREC.a.3.2.10a})\label{HAREC.a.3.2.10a}
  and T filter (\ssaref{myHAREC.a.3.2.10b})\label{HAREC.a.3.2.10b};
\item Quartz crystal; (\ssaref{myHAREC.a.3.2.11})\label{HAREC.a.3.2.11}
\item Effects due to real (=non-ideal) components;
  (\ssaref{myHAREC.a.3.2.12})\label{HAREC.a.3.2.12}
\item digital filters (see sections 1.10 and 3.8).
  (\ssaref{myHAREC.a.3.2.13})\label{HAREC.a.3.2.13}
\end{enumerate}

\item Power supply; (\ssaref{myHAREC.a.3.3})\label{HAREC.a.3.3}
\begin{enumerate}
\item Circuits for half-wave and full-wave rectification and the Bridge
  rectifier; (\ssaref{myHAREC.a.3.3.1})\label{HAREC.a.3.3.1}
\item Smoothing circuits; (\ssaref{myHAREC.a.3.3.2})\label{HAREC.a.3.3.2}
\item Stabilisation circuits in low voltage supplies;
  (\ssaref{myHAREC.a.3.3.3})\label{HAREC.a.3.3.3}
\item Switching mode power supplies, isolation and EMC.
  (\ssaref{myHAREC.a.3.3.4})\label{HAREC.a.3.3.4}
\end{enumerate}

\item Amplifier; (\ssaref{myHAREC.a.3.4})\label{HAREC.a.3.4}
\begin{enumerate}
\item Lf and hf amplifiers; (\ssaref{myHAREC.a.3.4.1})\label{HAREC.a.3.4.1}
\item Gain; (\ssaref{myHAREC.a.3.4.2})\label{HAREC.a.3.4.2}
\item Amplitude/frequency characteristic and bandwidth (broadband vs.
  tuned stages); (\ssaref{myHAREC.a.3.4.3})\label{HAREC.a.3.4.3}
\item Class A, A/B, B and C biasing;
  (\ssaref{myHAREC.a.3.4.4})\label{HAREC.a.3.4.4}
\item Harmonic and intermodulation distortion, overdriving amplifier stages.
  (\ssaref{myHAREC.a.3.4.5})\label{HAREC.a.3.4.5}
\end{enumerate}

\item Detector; (\ssaref{myHAREC.a.3.5})\label{HAREC.a.3.5}
\begin{enumerate}
\item AM detectors (envelope detectors);
  (\ssaref{myHAREC.a.3.5.1})\label{HAREC.a.3.5.1}
\item Diode detector; (\ssaref{myHAREC.a.3.5.2})\label{HAREC.a.3.5.2}
\item Product detectors and beat oscillators;
  (\ssaref{myHAREC.a.3.5.3})\label{HAREC.a.3.5.3}
\item FM detectors. (\ssaref{myHAREC.a.3.5.4})\label{HAREC.a.3.5.4}
\end{enumerate}

\item Oscillator; (\ssaref{myHAREC.a.3.6})\label{HAREC.a.3.6}
\begin{enumerate}
\item Feedback (intentional and unintentional oscillations);
  (\ssaref{myHAREC.a.3.6.1})\label{HAREC.a.3.6.1}
\item Factors affecting frequency and frequency stability conditions necessary
  for oscillation; (\ssaref{myHAREC.a.3.6.2})\label{HAREC.a.3.6.2}
\item LC oscillator; (\ssaref{myHAREC.a.3.6.3})\label{HAREC.a.3.6.3}
\item Crystal oscillator, overtone oscillator;
  (\ssaref{myHAREC.a.3.6.4})\label{HAREC.a.3.6.4}
\item Voltage controlled oscillator (VCO);
  (\ssaref{myHAREC.a.3.6.5})\label{HAREC.a.3.6.5}
\item Phase noise. (\ssaref{myHAREC.a.3.6.6})\label{HAREC.a.3.6.6}
\end{enumerate}

\item Phase Locked Loop [PLL]; (\ssaref{myHAREC.a.3.7})\label{HAREC.a.3.7}
\begin{enumerate}
\item Control loop with phase comparator circuit;
  (\ssaref{myHAREC.a.3.7.1})\label{HAREC.a.3.7.1}
\item Frequency synthesis with a programmable divider in the feedback loop.
  (\ssaref{myHAREC.a.3.7.2})\label{HAREC.a.3.7.2}
\end{enumerate}

\item Discrete Time Signals and Systems (DSP-systems).
  (\ssaref{myHAREC.a.3.8})\label{HAREC.a.3.8}
\begin{enumerate}
\item FIR and IIR filter topologies;
  (\ssaref{myHAREC.a.3.8.1})\label{HAREC.a.3.8.1}
\item Fourier Transformation (DFT; FFT, graphical presentation);
  (\ssaref{myHAREC.a.3.8.2})\label{HAREC.a.3.8.2}
\item Direct Digital Synthesis. (\ssaref{myHAREC.a.3.8.3})\label{HAREC.a.3.8.3}
\end{enumerate}
\end{enumerate}

\item Receivers
\begin{enumerate}

\item Types;
\begin{enumerate}
\item Single (\ssaref{myHAREC.a.4.1.1a})\label{HAREC.a.4.1.1a}
  and double (\ssaref{myHAREC.a.4.1.1b})\label{HAREC.a.4.1.1b}
  superheterodyne receiver;
\item Direct conversion receivers. (\ssaref{myHAREC.a.4.1.2})\label{HAREC.a.4.1.2}
\end{enumerate}

\item Block diagrams;
\begin{enumerate}
\item CW receiver [A1A]; (\ssaref{myHAREC.a.4.2.1})\label{HAREC.a.4.2.1}
\item AM receiver [A3E]; (\ssaref{myHAREC.a.4.2.2})\label{HAREC.a.4.2.2}
\item SSB receiver for suppressed carrier telephony [J3E];
  (\ssaref{myHAREC.a.4.2.3})\label{HAREC.a.4.2.3}
\item FM receiver [F3E]. (\ssaref{myHAREC.a.4.2.4})\label{HAREC.a.4.2.4}
\end{enumerate}

\item Operation and function of the following stages;
\begin{enumerate}
\item HF amplifier [with tuned or fixed band pass];
  (\ssaref{myHAREC.a.4.3.1})\label{HAREC.a.4.3.1}
\item Oscillator [fixed and variable];
  (\ssaref{myHAREC.a.4.3.2})\label{HAREC.a.4.3.2}
\item Mixer; (\ssaref{myHAREC.a.4.3.3})\label{HAREC.a.4.3.3}
\item Intermediate frequency amplifier;
  (\ssaref{myHAREC.a.4.3.4})\label{HAREC.a.4.3.4}
\item Limiter; (\ssaref{myHAREC.a.4.3.5})\label{HAREC.a.4.3.5}
\item Detector, including product detector;
  (\ssaref{myHAREC.a.4.3.6})\label{HAREC.a.4.3.6}
\item Audio amplifier; (\ssaref{myHAREC.a.4.3.7})\label{HAREC.a.4.3.7}
\item Automatic gain control; (\ssaref{myHAREC.a.4.3.8})\label{HAREC.a.4.3.8}
\item S meter; (\ssaref{myHAREC.a.4.3.9})\label{HAREC.a.4.3.9}
\item Squelch. (\ssaref{myHAREC.a.4.3.10})\label{HAREC.a.4.3.10}
\end{enumerate}

\item Receiver characteristics. (\ssaref{myHAREC.a.4.4})\label{HAREC.a.4.4}
\begin{enumerate}
\item Adjacent-channel; (\ssaref{myHAREC.a.4.4.1})\label{HAREC.a.4.4.1}
\item Selectivity; (\ssaref{myHAREC.a.4.4.2})\label{HAREC.a.4.4.2}
\item Sensitivity, receiver noise, noise figure;
  (\ssaref{myHAREC.a.4.4.3})\label{HAREC.a.4.4.3}
\item Stability; (\ssaref{myHAREC.a.4.4.4})\label{HAREC.a.4.4.4}
\item Image frequency; (\ssaref{myHAREC.a.4.4.5})\label{HAREC.a.4.4.5}
\item Desensitization / Blocking; (\ssaref{myHAREC.a.4.4.6})\label{HAREC.a.4.4.6}
\item Intermodulation; cross modulation;
  (\ssaref{myHAREC.a.4.4.7})\label{HAREC.a.4.4.7}
\item Reciprocal mixing [phase noise].
  (\ssaref{myHAREC.a.4.4.8})\label{HAREC.a.4.4.8}
\end{enumerate}
\end{enumerate}

\item Transmitters
\begin{enumerate}
\item Types;

\begin{enumerate}
\item Transmitter with (\ssaref{myHAREC.a.5.1.1a})\label{HAREC.a.5.1.1a}
  or without (\ssaref{myHAREC.a.5.1.1b})\label{HAREC.a.5.1.1b}
  frequency translation.
\end{enumerate}

\item Block diagrams;
\begin{enumerate}
\item CW transmitter [A1A]; (\ssaref{myHAREC.a.5.2.1})\label{HAREC.a.5.2.1}
\item SSB transmitter with suppressed carrier telephony [J3E];
  (\ssaref{myHAREC.a.5.2.2})\label{HAREC.a.5.2.2}
\item FM transmitter with the audio signal modulating the VCO of the PLL [F3E].
  (\ssaref{myHAREC.a.5.2.3})\label{HAREC.a.5.2.3}
\end{enumerate}

\item Operation and function of the following stages;
\begin{enumerate}
\item Mixer; (\ssaref{myHAREC.a.5.3.1})\label{HAREC.a.5.3.1}
\item Oscillator; (\ssaref{myHAREC.a.5.3.2})\label{HAREC.a.5.3.2}
\item Buffer; (\ssaref{myHAREC.a.5.3.3})\label{HAREC.a.5.3.3}
\item Driver; (\ssaref{myHAREC.a.5.3.4})\label{HAREC.a.5.3.4}
\item Frequency multiplier; (\ssaref{myHAREC.a.5.3.5})\label{HAREC.a.5.3.5}
\item Power amplifier; (\ssaref{myHAREC.a.5.3.6})\label{HAREC.a.5.3.6}
\item Output matching; (\ssaref{myHAREC.a.5.3.7})\label{HAREC.a.5.3.7}
\item Output filter; (\ssaref{myHAREC.a.5.3.8})\label{HAREC.a.5.3.8}
\item Frequency modulator; (\ssaref{myHAREC.a.5.3.9})\label{HAREC.a.5.3.9}
\item SSB modulator; (\ssaref{myHAREC.a.5.3.10})\label{HAREC.a.5.3.10}
\item Phase modulator; (\ssaref{myHAREC.a.5.3.11})\label{HAREC.a.5.3.11}
\item Crystal filter. (\ssaref{myHAREC.a.5.3.12})\label{HAREC.a.5.3.12}
\end{enumerate}

\item Transmitter characteristics. (\ssaref{myHAREC.a.5.4})\label{HAREC.a.5.4}
\begin{enumerate}
\item Frequency stability; (\ssaref{myHAREC.a.5.4.1})\label{HAREC.a.5.4.1}
\item RF-bandwidth; (\ssaref{myHAREC.a.5.4.2})\label{HAREC.a.5.4.2}
\item Sidebands; (\ssaref{myHAREC.a.5.4.3})\label{HAREC.a.5.4.3}
\item Audio-frequency range; (\ssaref{myHAREC.a.5.4.4})\label{HAREC.a.5.4.4}
\item Non-linearity [harmonic and intermodulation distortion];
  (\ssaref{myHAREC.a.5.4.5})\label{HAREC.a.5.4.5}
\item Output impedance; (\ssaref{myHAREC.a.5.4.6})\label{HAREC.a.5.4.6}
\item Output power; (\ssaref{myHAREC.a.5.4.7})\label{HAREC.a.5.4.7}
\item Efficiency; (\ssaref{myHAREC.a.5.4.8})\label{HAREC.a.5.4.8}
\item Frequency deviation; (\ssaref{myHAREC.a.5.4.9})\label{HAREC.a.5.4.9}
\item Modulation index; (\ssaref{myHAREC.a.5.4.10})\label{HAREC.a.5.4.10}
\item CW key clicks and chirps; (\ssaref{myHAREC.a.5.4.11})\label{HAREC.a.5.4.11}
\item SSB overmodulation and splatter (agreed);
  (\ssaref{myHAREC.a.5.4.12})\label{HAREC.a.5.4.12}
\item Spurious RF radiations (agreed);
  (\ssaref{myHAREC.a.5.4.13})\label{HAREC.a.5.4.13}
\item Cabinet radiations; (\ssaref{myHAREC.a.5.4.14})\label{HAREC.a.5.4.14}
\item Phase noise. (\ssaref{myHAREC.a.5.4.15})\label{HAREC.a.5.4.15}
\end{enumerate}
\end{enumerate}

\item Antennas and transmission lines
\begin{enumerate}

\item Antenna types;
\begin{enumerate}
\item Centre fed half-wave antenna; (\ssaref{myHAREC.a.6.1.1})\label{HAREC.a.6.1.1}
\item End fed half-wave antenna; (\ssaref{myHAREC.a.6.1.2})\label{HAREC.a.6.1.2}
\item Folded dipole; (\ssaref{myHAREC.a.6.1.3})\label{HAREC.a.6.1.3}
\item Quarter-wave vertical antenna [ground plane];
  (\ssaref{myHAREC.a.6.1.4})\label{HAREC.a.6.1.4}
\item Antenna with parasitic elements [Yagi];
  (\ssaref{myHAREC.a.6.1.5})\label{HAREC.a.6.1.5}
\item Aperture antennas (Parabolic reflector, horn);
  (\ssaref{myHAREC.a.6.1.6})\label{HAREC.a.6.1.6}
\item Trap dipole. (\ssaref{myHAREC.a.6.1.7})\label{HAREC.a.6.1.7}
\end{enumerate}

\item Antenna characteristics;
\begin{enumerate}
\item Distribution of the current and voltage;
  (\ssaref{myHAREC.a.6.2.1})\label{HAREC.a.6.2.1}
\item Impedance at the feed point;
  (\ssaref{myHAREC.a.6.2.2})\label{HAREC.a.6.2.2}
\item Capacitive or inductive impedance of a non-resonant antenna;
  (\ssaref{myHAREC.a.6.2.3})\label{HAREC.a.6.2.3}
\item Polarisation; (\ssaref{myHAREC.a.6.2.4})\label{HAREC.a.6.2.4}
\item Antenna directivity, efficiency and gain;
  (\ssaref{myHAREC.a.6.2.5})\label{HAREC.a.6.2.5}
\item Capture area; (\ssaref{myHAREC.a.6.2.6})\label{HAREC.a.6.2.6}
\item Radiated power [ERP, EIRP]; (\ssaref{myHAREC.a.6.2.7})\label{HAREC.a.6.2.7}
\item Front-to-back ratio; (\ssaref{myHAREC.a.6.2.8})\label{HAREC.a.6.2.8}
\item Horizontal and vertical radiation patterns.
  (\ssaref{myHAREC.a.6.2.9})\label{HAREC.a.6.2.9}
\end{enumerate}

\item Transmission lines.
\begin{enumerate}
\item Parallel conductor line; (\ssaref{myHAREC.a.6.3.1})\label{HAREC.a.6.3.1}
\item Coaxial cable; (\ssaref{myHAREC.a.6.3.2})\label{HAREC.a.6.3.2}
\item Waveguide; (\ssaref{myHAREC.a.6.3.3})\label{HAREC.a.6.3.3}
\item Characteristic impedance [Z0];
  (\ssaref{myHAREC.a.6.3.4})\label{HAREC.a.6.3.4}
\item Velocity factor; (\ssaref{myHAREC.a.6.3.5})\label{HAREC.a.6.3.5}
\item Standing-wave ratio; (\ssaref{myHAREC.a.6.3.6})\label{HAREC.a.6.3.6}
\item Losses; (\ssaref{myHAREC.a.6.3.7})\label{HAREC.a.6.3.7}
\item Balun; (\ssaref{myHAREC.a.6.3.8})\label{HAREC.a.6.3.8}
\item Antenna tuning units (pi and T configurations only).
  (\ssaref{myHAREC.a.6.3.9})\label{HAREC.a.6.3.9}
\end{enumerate}
\end{enumerate}

\item Propagation
\begin{enumerate}
\item Signal attenuation (\ssaref{myHAREC.a.7.1.1})\label{HAREC.a.7.1.1},
  signal to noise ratio (\ssaref{myHAREC.a.7.1.2})\label{HAREC.a.7.1.2};
\item Line of sight propagation (free space propagation, inverse square law);
  (\ssaref{myHAREC.a.7.2})\label{HAREC.a.7.2}
\item Ionospheric layers; (\ssaref{myHAREC.a.7.3})\label{HAREC.a.7.3}
\item Critical frequency; (\ssaref{myHAREC.a.7.4})\label{HAREC.a.7.4}
\item Influence of the sun on the ionosphere;
  (\ssaref{myHAREC.a.7.5})\label{HAREC.a.7.5}
\item Maximum Usable Frequency; (\ssaref{myHAREC.a.7.6})\label{HAREC.a.7.6}
\item Ground wave and sky wave, angle of radiation and skip distance;
  (\ssaref{myHAREC.a.7.7})\label{HAREC.a.7.7}
\item Multipath in ionospheric propagation;
  (\ssaref{myHAREC.a.7.8})\label{HAREC.a.7.8}
\item Fading; (\ssaref{myHAREC.a.7.9})\label{HAREC.a.7.9}
\item Troposphere (Ducting, scattering);
  (\ssaref{myHAREC.a.7.10})\label{HAREC.a.7.10}
\item The influence of the height of antennas on the distance that can be
  covered [radio horizon]; (\ssaref{myHAREC.a.7.11})\label{HAREC.a.7.11}
\item Temperature inversion; (\ssaref{myHAREC.a.7.12})\label{HAREC.a.7.12}
\item Sporadic E-reflection; (\ssaref{myHAREC.a.7.13})\label{HAREC.a.7.13}
\item Auroral scattering; (\ssaref{myHAREC.a.7.14})\label{HAREC.a.7.14}
\item Meteor scatter; (\ssaref{myHAREC.a.7.15})\label{HAREC.a.7.15}
\item Reflections from the moon; (\ssaref{myHAREC.a.7.16})\label{HAREC.a.7.16}
\item Atmospheric noise [distant thunderstorms];
  (\ssaref{myHAREC.a.7.17})\label{HAREC.a.7.17}
\item Galactic noise; (\ssaref{myHAREC.a.7.18})\label{HAREC.a.7.18}
\item Ground (thermal) noise. (\ssaref{myHAREC.a.7.19})\label{HAREC.a.7.19}
\item Propagation prediction basics (link budget):
  (\ssaref{myHAREC.a.7.20})\label{HAREC.a.7.20}
\begin{enumerate}
\item dominant noise source, (band noise vs. receiver noise);
  (\ssaref{myHAREC.a.7.20.1})\label{HAREC.a.7.20.1}
\item minimum signal to noise ratio;
  (\ssaref{myHAREC.a.7.20.2})\label{HAREC.a.7.20.2}
\item minimum received signal power;
  (\ssaref{myHAREC.a.7.20.3})\label{HAREC.a.7.20.3}
\item path loss;
  (\ssaref{myHAREC.a.7.20.4})\label{HAREC.a.7.20.4}
\item antenna gains, transmission line losses;
  (\ssaref{myHAREC.a.7.20.5})\label{HAREC.a.7.20.5}
\item minimum transmitter power.
  (\ssaref{myHAREC.a.7.20.6})\label{HAREC.a.7.20.6}
\end{enumerate}
\end{enumerate}

\item Measurements
\begin{enumerate}

\item Making measurements; (\ssaref{myHAREC.a.8.1})\label{HAREC.a.8.1}
\begin{enumerate}
\item Measurement of:
\begin{enumerate}
\item DC and AC voltages and currents;
  (\ssaref{myHAREC.a.8.1.1.1})\label{HAREC.a.8.1.1.1}
\end{enumerate}
\item Measuring errors:
\begin{enumerate}
\item Influence of frequency; (\ssaref{myHAREC.a.8.1.2.1})\label{HAREC.a.8.1.2.1}
\item Influence of waveform; (\ssaref{myHAREC.a.8.1.2.2})\label{HAREC.a.8.1.2.2}
\item Influence of internal resistance of meters.
  (\ssaref{myHAREC.a.8.1.2.3})\label{HAREC.a.8.1.2.3}
\end{enumerate}
\item Resistance; (\ssaref{myHAREC.a.8.1.3})\label{HAREC.a.8.1.3}
\item DC and RF power [average power, Peak Envelope Power];
  (\ssaref{myHAREC.a.8.1.4})\label{HAREC.a.8.1.4}
\item Voltage standing-wave ratio; (\ssaref{myHAREC.a.8.1.5})\label{HAREC.a.8.1.5}
\item Waveform of the envelope of an RF signal;
  (\ssaref{myHAREC.a.8.1.6})\label{HAREC.a.8.1.6}
\item Frequency; (\ssaref{myHAREC.a.8.1.7})\label{HAREC.a.8.1.7}
\item Resonant frequency. (\ssaref{myHAREC.a.8.1.8})\label{HAREC.a.8.1.8}
\end{enumerate}

\item Measuring instruments. (\ssaref{myHAREC.a.8.2})\label{HAREC.a.8.2}
\begin{enumerate}
\item Making measurements using:
\begin{enumerate}
\item Multi range meter (digital and analog);
  (\ssaref{myHAREC.a.8.2.1.1})\label{HAREC.a.8.2.1.1}
\item Rf-power meter; (\ssaref{myHAREC.a.8.2.1.2})\label{HAREC.a.8.2.1.2}
\item Reflectometer bridge (SWR meter);
  (\ssaref{myHAREC.a.8.2.1.3})\label{HAREC.a.8.2.1.3}
\item Signal generator; (\ssaref{myHAREC.a.8.2.1.4})\label{HAREC.a.8.2.1.4}
\item Frequency counter; (\ssaref{myHAREC.a.8.2.1.5})\label{HAREC.a.8.2.1.5}
\item Oscilloscope; (\ssaref{myHAREC.a.8.2.1.6})\label{HAREC.a.8.2.1.6}
\item Spectrum Analyzer. (\ssaref{myHAREC.a.8.2.1.7})\label{HAREC.a.8.2.1.7}
\end{enumerate}
\end{enumerate}
\end{enumerate}

\item Interference and immunity
\begin{enumerate}

\item Interference in electronic equipment;
\begin{enumerate}
\item Blocking (\ssaref{myHAREC.a.9.1.1})\label{HAREC.a.9.1.1}
\item Interference with the desired signal
  (\ssaref{myHAREC.a.9.1.2})\label{HAREC.a.9.1.2}
\item Intermodulation (\ssaref{myHAREC.a.9.1.3})\label{HAREC.a.9.1.3}
\item Detection in audio circuits (\ssaref{myHAREC.a.9.1.4})\label{HAREC.a.9.1.4}
\end{enumerate}

\item Cause of interference in electronic equipment;
\begin{enumerate}
\item Field strength of the transmitter
  (\ssaref{myHAREC.a.9.2.1})\label{HAREC.a.9.2.1}
\item Spurious radiation of the transmitter [parasitic radiation, harmonics]
  (\ssaref{myHAREC.a.9.2.2})\label{HAREC.a.9.2.2}
\item Undesired influence on the equipment:
\begin{enumerate}
\item via the antenna input [aerial voltage, input selectivity]
  (\ssaref{myHAREC.a.9.2.3.1})\label{HAREC.a.9.2.3.1}
\item via other connected lines
  (\ssaref{myHAREC.a.9.2.3.2})\label{HAREC.a.9.2.3.2}
\item by direct radiation
  (\ssaref{myHAREC.a.9.2.3.3})\label{HAREC.a.9.2.3.3}
\end{enumerate}
\end{enumerate}

\item Measures against interference.
\begin{enumerate}
\item Measures to prevent and eliminate interference effects:
\begin{enumerate}
\item Filtering (\ssaref{myHAREC.a.9.3.1.1})\label{HAREC.a.9.3.1.1}
\item Decoupling (\ssaref{myHAREC.a.9.3.1.2})\label{HAREC.a.9.3.1.2}
\item Shielding (\ssaref{myHAREC.a.9.3.1.3})\label{HAREC.a.9.3.1.3}
\end{enumerate}
\end{enumerate}
\end{enumerate}

\item SAFETY
\begin{enumerate}
\item The human body (\ssaref{myHAREC.a.10.1})\label{HAREC.a.10.1}
\item Mains power supply (\ssaref{myHAREC.a.10.2})\label{HAREC.a.10.2}
\item High voltages (\ssaref{myHAREC.a.10.3})\label{HAREC.a.10.3}
\item Lightning (\ssaref{myHAREC.a.10.4})\label{HAREC.a.10.4}
\end{enumerate}
\end{enumerate}

\section[Rules and procedures]{National and international operating rules and procedures}

\begin{enumerate}

\item Phonetic Alphabet
\begin{enumerate}
\item A = Alpha (\ssaref{myHAREC.b.1.1})\label{HAREC.b.1.1}
\item B = Bravo (\ssaref{myHAREC.b.1.1})\label{HAREC.b.1.2}
\item C = Charlie (\ssaref{myHAREC.b.1.1})\label{HAREC.b.1.3}
\item D = Delta (\ssaref{myHAREC.b.1.1})\label{HAREC.b.1.4}
\item E = Echo (\ssaref{myHAREC.b.1.1})\label{HAREC.b.1.5}
\item F = Foxtrot (\ssaref{myHAREC.b.1.1})\label{HAREC.b.1.6}
\item G = Golf (\ssaref{myHAREC.b.1.1})\label{HAREC.b.1.7}
\item H = Hotel (\ssaref{myHAREC.b.1.1})\label{HAREC.b.1.8}
\item I = India (\ssaref{myHAREC.b.1.1})\label{HAREC.b.1.9}
\item J = Juliett (\ssaref{myHAREC.b.1.1})\label{HAREC.b.1.10}
\item K = Kilo (\ssaref{myHAREC.b.1.1})\label{HAREC.b.1.11}
\item L = Lima (\ssaref{myHAREC.b.1.1})\label{HAREC.b.1.12}
\item M = Mike (\ssaref{myHAREC.b.1.1})\label{HAREC.b.1.13}
\item N = November (\ssaref{myHAREC.b.1.1})\label{HAREC.b.1.14}
\item O = Oscar (\ssaref{myHAREC.b.1.1})\label{HAREC.b.1.15}
\item P = Papa (\ssaref{myHAREC.b.1.1})\label{HAREC.b.1.16}
\item Q = Quebec (\ssaref{myHAREC.b.1.1})\label{HAREC.b.1.17}
\item R = Romeo (\ssaref{myHAREC.b.1.1})\label{HAREC.b.1.18}
\item S = Sierra (\ssaref{myHAREC.b.1.1})\label{HAREC.b.1.19}
\item T = Tango (\ssaref{myHAREC.b.1.1})\label{HAREC.b.1.20}
\item U = Uniform (\ssaref{myHAREC.b.1.1})\label{HAREC.b.1.21}
\item V = Victor (\ssaref{myHAREC.b.1.1})\label{HAREC.b.1.22}
\item W = Whiskey (\ssaref{myHAREC.b.1.1})\label{HAREC.b.1.23}
\item X = X-ray (\ssaref{myHAREC.b.1.1})\label{HAREC.b.1.24}
\item Y = Yankee (\ssaref{myHAREC.b.1.1})\label{HAREC.b.1.25}
\item Z = Zulu (\ssaref{myHAREC.b.1.1})\label{HAREC.b.1.26}
\end{enumerate}

\item Q-Code
\begin{enumerate}
\item QRK? = What is the readability of my signals?
  (\ssaref{myHAREC.b.2.1})\label{HAREC.b.2.1}
\item QRK  = The readability of your signals is ...
  (\ssaref{myHAREC.b.2.1})\label{HAREC.b.2.2}
\item QRM? = Are you being interfered with?
  (\ssaref{myHAREC.b.2.1})\label{HAREC.b.2.3}
\item QRM  = I am being interfered with ...
  (\ssaref{myHAREC.b.2.1})\label{HAREC.b.2.4}
\item QRN? = Are you troubled by static?
  (\ssaref{myHAREC.b.2.1})\label{HAREC.b.2.5}
\item QRN  = I am troubled by static
  (\ssaref{myHAREC.b.2.1})\label{HAREC.b.2.6}
\item QRO? = Shall I increase transmitter power?
  (\ssaref{myHAREC.b.2.1})\label{HAREC.b.2.7}
\item QRO  = Increase transmitter power
  (\ssaref{myHAREC.b.2.1})\label{HAREC.b.2.8}
\item QRP? = Shall I decrease transmitter power?
  (\ssaref{myHAREC.b.2.1})\label{HAREC.b.2.9}
\item QRP  = Decrease transmitter power
  (\ssaref{myHAREC.b.2.1})\label{HAREC.b.2.10}
\item QRT? = Shall I stop sending?
  (\ssaref{myHAREC.b.2.1})\label{HAREC.b.2.11}
\item QRT  = Stop sending
  (\ssaref{myHAREC.b.2.1})\label{HAREC.b.2.12}
\item QRZ? = Who is calling me?
  (\ssaref{myHAREC.b.2.1})\label{HAREC.b.2.13}
\item QRZ  = You are being called by ...
  (\ssaref{myHAREC.b.2.1})\label{HAREC.b.2.14}
\item QRV? = Are you ready?
  (\ssaref{myHAREC.b.2.1})\label{HAREC.b.2.15}
\item QRV  = I am ready
  (\ssaref{myHAREC.b.2.1})\label{HAREC.b.2.16}
\item QSB? = Are my signals fading?
  (\ssaref{myHAREC.b.2.1})\label{HAREC.b.2.17}
\item QSB  = Your signals are fading
  (\ssaref{myHAREC.b.2.1})\label{HAREC.b.2.18}
\item QSL? = Can you acknowledge receipt?
  (\ssaref{myHAREC.b.2.1})\label{HAREC.b.2.19}
\item QSL  = I am acknowledging receipt
  (\ssaref{myHAREC.b.2.1})\label{HAREC.b.2.20}
\item QSO? = Can you communicate with ... direct?
  (\ssaref{myHAREC.b.2.1})\label{HAREC.b.2.21}
\item QSO  = I can communicate ... direct
  (\ssaref{myHAREC.b.2.1})\label{HAREC.b.2.22}
\item QSY? = Shall I change to transmission on another frequency?
  (\ssaref{myHAREC.b.2.1})\label{HAREC.b.2.23}
\item QSY  = Change transmission to another frequency
  (\ssaref{myHAREC.b.2.1})\label{HAREC.b.2.24}
\item QRX? = When will you call again?
  (\ssaref{myHAREC.b.2.1})\label{HAREC.b.2.25}
\item QRX  = I will call you again at ... hours on ... kHz (or MHz)
  (\ssaref{myHAREC.b.2.1})\label{HAREC.b.2.26}
\item QTH? = What is your position in latitude and longitude (or according to
  any other indication)? (\ssaref{myHAREC.b.2.1})\label{HAREC.b.2.27}
\item QTH  = My position is  ... latitude, ... longitude (or according to any
  other indication) (\ssaref{myHAREC.b.2.1})\label{HAREC.b.2.28}
\end{enumerate}

\item Operational Abbreviations
\begin{enumerate}
\item BK = Signal used to interrupt a transmission in progress
  (\ssaref{myHAREC.b.3.1})\label{HAREC.b.3.1}
\item CQ = General call to all stations
  (\ssaref{myHAREC.b.3.1})\label{HAREC.b.3.2}
\item CW = Continuous wave
  (\ssaref{myHAREC.b.3.1})\label{HAREC.b.3.3}
\item DE = From, used to separate the call sign of the station called from that
  of the calling station (\ssaref{myHAREC.b.3.1})\label{HAREC.b.3.4}
\item K = Invitation to transmit
  (\ssaref{myHAREC.b.3.1})\label{HAREC.b.3.5}
\item MSG = Message
  (\ssaref{myHAREC.b.3.1})\label{HAREC.b.3.6}
\item PSE = Please
  (\ssaref{myHAREC.b.3.1})\label{HAREC.b.3.7}
\item RST = Readability, signal-strength, tone-report
  (\ssaref{myHAREC.b.3.1})\label{HAREC.b.3.8}
\item R = Received
  (\ssaref{myHAREC.b.3.1})\label{HAREC.b.3.9}
\item RX = Receiver
  (\ssaref{myHAREC.b.3.1})\label{HAREC.b.3.10}
\item TX = Transmitter
  (\ssaref{myHAREC.b.3.1})\label{HAREC.b.3.11}
\item UR = Your
  (\ssaref{myHAREC.b.3.1})\label{HAREC.b.3.12}
\end{enumerate}

\item International distress signs, emergency traffic and natural disaster
  communication
\begin{enumerate}
\item radiotelegraph ...---... [SOS] (\ssaref{myHAREC.b.4.1})\label{HAREC.b.4.1}
\item radiotelephone ''MAYDAY'' (\ssaref{myHAREC.b.4.1})\label{HAREC.b.4.2}
\item International use of the amateur station in the event of national
  disasters; (\ssaref{myHAREC.b.4.1})\label{HAREC.b.4.3}
\item Frequency bands allocated to the amateur service and amateur satellite
  service. (\ssaref{myHAREC.b.4.1})\label{HAREC.b.4.4}
\end{enumerate}

\item Call signs
\begin{enumerate}
\item Identification of the amateur station;
  (\ssaref{myHAREC.b.5.1})\label{HAREC.b.5.1}
\item Use of the call signs; (\ssaref{myHAREC.b.5.2})\label{HAREC.b.5.2}
\item Composition of call signs; (\ssaref{myHAREC.b.5.3})\label{HAREC.b.5.3}
\item National prefixes. (\ssaref{myHAREC.b.5.4})\label{HAREC.b.5.4}
\end{enumerate}

\item IARU band plans
\begin{enumerate}
\item IARU band plans; (\ssaref{myHAREC.b.6.1})\label{HAREC.b.6.1}
\item Purposes. (\ssaref{myHAREC.b.6.2})\label{HAREC.b.6.2}
\end{enumerate}

\item Social responsibility and operating procedures
\begin{enumerate}
\item Social responsibility of radio amateur operation
\begin{enumerate}
\item The Radio Amateur Code of Conduct;
  (\ssaref{myHAREC.b.7.1.1})\label{HAREC.b.7.1.1}
\item Self-regulation and self-discipline in Amateur Radio.
  (\ssaref{myHAREC.b.7.1.2})\label{HAREC.b.7.1.2}
\end{enumerate}

\item Operating procedures
\begin{enumerate}
\item Starting, carrying out and ending a contact;
  (\ssaref{myHAREC.b.7.2.1})\label{HAREC.b.7.2.1}
\item Correct use of call signs and abbreviations;
  (\ssaref{myHAREC.b.7.2.2})\label{HAREC.b.7.2.2}
\item Content of transmissions;
  (\ssaref{myHAREC.b.7.2.3})\label{HAREC.b.7.2.3}
\item Checking transmission quality.
  (\ssaref{myHAREC.b.7.2.4})\label{HAREC.b.7.2.4}
\end{enumerate}
\end{enumerate}

\end{enumerate}

\newpage

\section[Regulations]{National and international regulations relevant to the amateur service and amateur satellite service}

\begin{enumerate}

\item ITU radio regulations
\begin{enumerate}
\item Definition Amateur Service and Amateur Satellite Service;
  (\ssaref{myHAREC.c.1.1})\label{HAREC.c.1.1}
\item Definition Amateur station;
  (\ssaref{myHAREC.c.1.2})\label{HAREC.c.1.2}
\item Article 25 Radio Regulations;
  (\ssaref{myHAREC.c.1.3})\label{HAREC.c.1.3}
\item Status Amateur Service and Amateur Satellite Service;
  (\ssaref{myHAREC.c.1.4})\label{HAREC.c.1.4}
\item ITU Radio Regions.
  (\ssaref{myHAREC.c.1.5})\label{HAREC.c.1.5}
\end{enumerate}

\item CEPT regulations
\begin{enumerate}
\item Recommendation T/R 61-01;
  (\ssaref{myHAREC.c.2.1})\label{HAREC.c.2.1}
\item Temporary use of amateur stations in CEPT countries;
  (\ssaref{myHAREC.c.2.2})\label{HAREC.c.2.2}
\item Temporary use of amateur stations in NON-CEPT countries which participate
  in the T/R 61-01 system. (\ssaref{myHAREC.c.2.3})\label{HAREC.c.2.3}
\end{enumerate}

\item National laws, regulations and licence conditions
\begin{enumerate}
\item National laws
  (\ssaref{myHAREC.c.3.1})\label{HAREC.c.3.1}
\item Regulations and licence conditions
  (\ssaref{myHAREC.c.3.2})\label{HAREC.c.3.2}
\item Demonstrate knowledge of maintaining a log:
\begin{enumerate}
\item log keeping; (\ssaref{myHAREC.c.3.3.1})\label{HAREC.c.3.3.1}
\item purpose; (\ssaref{myHAREC.c.3.3.2})\label{HAREC.c.3.3.2}
\item recorded data. (\ssaref{myHAREC.c.3.3.3})\label{HAREC.c.3.3.3}
\end{enumerate}
\end{enumerate}
\end{enumerate}

%% k7per
\end{flushleft}


\onecolumn
\bibliography{koncept.bib}
\bibliographystyle{plain}

\backmatter

\end{document}
