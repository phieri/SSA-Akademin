\chapter{Måttenheter}

\noindent Inom fysiken förekommer allt mellan mycket höga och mycket
låga värden på frekvens, spänning, ström, resistans etc.
I en radiomottagares antenningång är signalspänningen ofta mindre än
\qty{0,000001}{\volt}.
I slutsteget i en amatörradiosändare kan anodspänningen vara högre än
\qty{2000}{\volt} och uteffekten upp till \qty{1000}{\watt}.
I spektrum för elektromagnetiska vågor finns mycket höga frekvenser så som
\qty{10000000000}{\hertz}.

För att ange storheten på måttenheter används ofta ett \emph{prefix}
före måttenheten (av latinets \emph{pre}, före och \emph{fixare}, att
tillägga).
Med prefixet anges vilken multiplikations- eller divisionsfaktor (talfaktor)
som används. Se tabell~\ssaref{tab:prefix}.

Prefix kan vid behov användas till alla enheter.
I nedanstående tabell används som exempel prefix tillsammans med enheterna
\unit{\hertz}, \unit{\watt}, \unit{\volt}, \unit{\farad} etc. som exempel.


\begin{table*}[b]
%  \begin{center}
    \begin{tabular}{|Sllll|}
      \hline
%      1 000 000 000 000 000 000 & \unit{\hertz} & = \(1 \cdot 10^{18}\)\,\unit{\hertz} & = \qty{1}{\exa\hertz} & (\unit{\exa\noop} uttalas exa) \\
%      1 000 000 000 000 000 & \unit{\hertz}     & = \(1 \cdot 10^{15}\)\,\unit{\hertz} & = \qty{1}{\peta\hertz}& (\unit{peta\noop} uttalas peta) \\
      1 000 000 000 000 & \unit{\hertz}         & = \(1 \cdot 10^{12}\)\,\unit{\hertz}  & = \qty{1}{\tera\hertz}& (\unit{\tera\noop} uttalas tera) \\
      1 000 000 000  & \unit{\watt}             & = \(1 \cdot 10^9\)\,\unit{\watt}     & = \qty{1}{\giga\watt} & (\unit{\giga\noop} uttalas giga) \\
      1 000 000  & \unit{\watt}                 & = \(1 \cdot 10^6\)\,\unit{\watt}              & = \qty{1 }{\mega\watt}& (\unit{\mega\noop} uttalas mega) \\
      1 000  & \unit{\watt}                     & = \(1 \cdot 10^3\)\,\unit{\watt}   & = \qty{1}{\kilo\watt} & (\unit{\kilo\noop} uttalas kilo) \\
      100 &  & = \(1 \cdot 10^2\)             &      & (\unit{\hecto\noop} uttalas hekto) \\
      10 & & = \(1 \cdot 10^1\) & & (\unit{\deca\noop} uttalas deka) \\
      1 & \unit{\volt} &  = \(1 \cdot 10^0\) V & \qty{1}{\volt} & (\(1 = 10^0\) är grundenhet) \\
      0,1 & \unit{\metre} & = \(1 \cdot 10^{-1}\) & \qty{1}{\deci\metre}    & (\unit{\deci\noop} uttalas deci) \\
      0,01 & \unit{\metre} & = \(1 \cdot 10^{-2}\) & \qty{1}{\centi\metre} & (\unit{\centi\noop} uttalas centi) \\
      0,001  & \unit{\volt}      & = \(1 \cdot 10^{-3}\)\,\unit{\volt}      & = \qty{1}{\milli\volt}      & (\unit{\milli\noop} uttalas milli) \\
      0,000 001 & \unit{\volt}      & = \(1 \cdot 10^{-6}\)\,\unit{\volt}      & = \qty{1}{\micro\volt}      & (\unit{\micro\noop} uttalas mikro) \\
      0,000 000 001 & \unit{\farad}      & = \(1 \cdot 10^{-9}\)\,\unit{\farad}& = \qty{1}{\nano\farad}      & (\unit{\nano\noop} uttalas nano) \\
      0,000 000 000 001  & \unit{F}      & = \(1 \cdot 10^{-12}\)\,\unit{\farad}      & = \qty{1}{\pico\farad}      & (\unit{\pico\noop} uttalas piko)\\
%      0,000 000 000 000 001 & \unit{\coulomb}      & = \(1 \cdot 10^{-15}\)\,\unit{\coulomb}  & = \qty{1}{\femto\coulomb}      & (\unit{\femto\noop} uttalas femto)\\
%      0,000 000 000 000 000 001  & \unit{\coulomb}      & = \(1 \cdot 10^{-18}\)\,\unit{\coulomb}      & = \qty{1}{\atto\coulomb}      & (\unit{\atto\noop} uttalas atto) \\
      \hline
    \end{tabular}
%  \end{center}
%  \caption{Prefix med några typiska måttenheter som exempel.}
  \label{tab:prefix}
\end{table*}

I exemplen ovan skrivs signalspänningen \qty{1}{\micro\volt}, anodspänningen
\qty{2}{\kilo\volt}, uteffekten \qty{1}{\kilo\watt} samt frekvensen
\qty{10}{\giga\hertz} vilket i många fall kan vara lättare att läsa och svårare
att misstolka.

Exponenter, till exempel siffran 6 i uttrycket \(10^6\), förklaras i
bilaga~\ssaref{potenser}.


\section{Flyttal}

En decimal talstorhet uttrycks ofta med ett så kallat \emph{tekniskt flyttal}.
Decimaltecknet placeras då så att den visade tio-exponenten i talet blir en
multipel av 3, som visas i \ssaref{tab:prefix}.

Decimaltecknet kan även placeras så att tioexponenten är något annat än en
multipel av 3.
Talstorheten uttrycks då med ett så kallat \emph{allmänt flyttal}.

I bland annat miniräknare visas ofta exponenten som bokstaven E, åtföljt av ett
värde.
Ibland utelämnas själva bokstaven medan exponentvärdet står kvar.

\bigskip

\noindent\begin{tabular}{Sllll}
1000  & blir & \(1    \cdot 10^3  \) & eller & 1 E+03 \\
125   & blir & \(1,25 \cdot 10^2  \) & eller & 1,25 E+02 \\
10    & blir & \(1    \cdot 10^1  \) & eller & 1 E+01 \\
1     & blir & \(1    \cdot 10^0  \) & eller & 1 E+00 \\
0,1   & blir & \(1    \cdot 10^{-1}\) & eller & 1 E-01 \\
0,01  & blir & \(1    \cdot 10^{-2}\) & eller & 1 E-02 \\
0,012 & blir & \(12   \cdot 10^{-3}\) & eller & 12 E-03 \\
\end{tabular}

\bigskip

Den klarsynte ser att vi i tabellen har slarvat med nogrannhet och att
vissa av talen ovan är avrundade till olika grad, i nästa stycke reder
vi ut begreppen kring noggrannhet.

\section{Närmevärden och Noggrannhet}

När vi skriver ett tal så är det oftast inte exakt det \emph{korrekta värdet},
vi skriver ner ett \emph{närmevärde} eller en \emph{approximation}.
Det kan vara för att vi inte vet exakt vad talet är, att vi bara kunnat mäta det
med en vis noggrannhet, eller att vi helt enkelt inte behöver mer än en viss
noggrannhet.
Antalet siffror, \emph{värdesiffror} eller \emph{signifikanta siffror}, vi
använder visar hur noggrant närmevärdet är.
Antalet värdesiffror är lika med antalet siffror i talet, exklusive inledande
nollor.
Om avslutande nollor är signifikanta eller inte beror på hur närmevärdet är
avrundat, se tabellen nedan för exempel.

\bigskip
\begin{centering}
\begin{tabular}{|l|l|}
\hline
Tal & Antal värdesiffror \\
\hline
0,04711 & 4\\
4711 & 4 \\
4711,000 & 7 \\
4\,711\,000 & 4 till 7 beroende på avrundning \\
\hline
\end{tabular}
\end{centering}
\bigskip

Den sista raden i tabellen visar problemet att storheten och noggrannheten inte
kan separeras om man skriver ett tal på vanligt vis.
Det finns inget sätt att veta vilka av de avslutande nollorna som är
värdesiffror och hur talet är avrundat.
För att inte blanda ihop storhet och noggrannhet hos ett tal använder man
\emph{vetenskaplig notation}, gärna med prefix som beskrevs ovan.
% Heter det grundpotensform?

Vetenskaplig notation skriver bara ut värdesiffrorna, med ett
decimalkomma efter den första värdesiffran, multiplicerat med en
tiopotens som bestämmer storheten.
Till exempel ljusets hastighet avrundat till 2 värdesiffror är
$3,0 \cdot 10^8$\,\unit{\metre\per\second}.

Om $|\Delta a|\leq 0,5\cdot 10^{-t}$, där $\Delta a$ är skillnaden mellan det
korrekta värdet och närmevärdet, sägs närmevärdet $\bar {a}$ ha $t$ korrekta
decimaler.
I ett närmevärde med $t>0$ korrekta decimaler sägs alla siffror i
positioner med enhet större än eller lika med $10^{-t}$ vara
\emph{signifikanta siffror}, utom inledande nollor, som endast anger
decimaltecknets läge.

Om man utför beräkningar med tal med olika antal värdesiffror så är en
bra tumregel att man i slutet har ett svar med lika många värdesiffror som i det
minst noggranna talet.
En mer detaljerad analys av hur beräkningar påverkar noggrannheten studeras inom
\emph{numerisk analys}.

Det är praktiskt viktigt att man inte använder fler värdesiffror än
man behöver.
De flesta mätningar man kan göra levererar ganska få värdesiffror.
För att sätta noggrannhet i perspektiv så påstås NASA använda 15 siffror för att
skicka rymdfarkoster runt i solsystemet och det behövs ungefär 39--40
värdesiffror för att beskriva universums omkrets till en atoms storlek.

%% \section{Metallers resistivitet}
%% \label{metallersresistivitet}

%% \begin{tabular}{l|l}
%%   Ämne & Resistivitet vid \qty{20}{\degreeCelsius} \([\dfrac{\unit{\ohm} \cdot mm^2}{m}]\) \\
%%   \hline
%%   Aluminium   & 0,028 \\
%%   Bly         & 0,22  \\
%%   Guld        & 0,024 \\
%%   Järn        & 0,105 \\
%%   Koppar      & 0,018 \\
%%   Kvicksilver & 0,958 \\
%%   Nickel      & 0,078 \\
%%   Platina     & 0,108 \\
%%   Silver      & 0,016 \\
%%   Tenn        & 0,115 \\
%%   Volfram     & 0,056 \\
%%   Zink        & 0,058 \\
%% \end{tabular}


\newpage
\section{Grekiska alfabetet}

Bokstäver ur bland annat grekiska alfabetet används som symboler för
tekniska begrepp.

Märk att samma symboler används olika inom olika teknikområden.
Nedan visas stora (versaler) och små (gemener) bokstäver, uttal, och dessutom
anges några vanliga användningar inom elektroniken.

\bigskip
%\begin{table*}
%  \begin{center}
  \begin{tabular}{ll|l|l}
    %    Stora  & Små   & Uttal & Exempel \\
        &    & Uttal & Exempel \\
    \hline
    \(A\) & \(\alpha\) & Alpha & \\
    \(B\) & \(\beta\) & Beta & \\
    \(\Gamma\) & \(\gamma\) & Gamma & Ledningsförmåga \\
    \(\Delta\) & & Delta & Del av .. storhet \\
    & \(\delta\) & Delta & Förlustvinkel etc. \\
    \(E\) & \(\varepsilon\) & Epsilon & Dielektricitets-\\
    & & & konstant etc.\\
    \(Z\) & \(\zeta\) & Zeta & \\
    \(H\) & \(\eta\) & \AE ta & Verkningsgrad\\
    \(\Theta\) & \(\vartheta\) & Teta & Vinklar \\
    \(I\) & \(\iota\) & Jota & \\
    \(K\) & \(\kappa\) & Kappa & Kopplingskoefficient \\
    \(\Lambda\) & \(\lambda\) & Lambda & Våglängd \\
    \(M\) & \(\mu\) & My & Permeabilitet \\
    \(N\) & \(\nu\) & Ny & Frekvens \\
    \(\Xi\) & \(\xi\) & Xi & \\
    \(O\) & \(o\) & Omikron & \\
    \(\Pi\) & \(\pi\) & Pi & 3,14159\dots \\
    \(P\) & \(\rho\) & Rho & Resistivitet \\
    \(\Sigma\) & \(\sigma\) & Sigma & Summa \\
    \(T\) & \(\tau\) & Tau & Tidskonstant \\
    \(Y\) & \(\upsilon\) & Ypsilon &  \\
    \(\Phi\) & & Fi & Magnetiskt flöde \\
    & \(\varphi\) & Fi & Fasvinkel \\
    \(X\) & \(\chi\) & Chi & \\
    \(\Psi\) & \(\psi\) & Psi & \\
    \(\Omega\) & & Omega & Resistans \\
    & \(\omega\) & Omega & Vinkelfrekvens \\
  \end{tabular}
%  \caption{Grekiska alfabetet med exempel på användning i
%    vetenskapliga och elektroniska sammanhang.}
%  \label{tab:a.grekiska}
%  \end{center}
%\end{table*}
