\newpage
\section[Ordningsregler]{Radioamatörens ordnings\-regler}
\harecsection{\harec{b}{7.1.2}{7.1.2}}

\subsection{Grundläggande principer}
\textbf{Grundläggande principer} som ska styra vårt \textbf{uppträdande} på
amatörbanden är:

\begin{description}
\item[Samhörighet, broderskap och kompiskänsla:] många, många av oss
  är aktiva i etern (vår spelplan).
  Vi är aldrig ensamma.
  Alla andra amatörer är våra kollegor, våra bröder och systrar, våra vänner.
  Agera därefter.
  Var alltid hänsynsfull.

\item[Tolerans:] inte alla amatörer delar nödvändigtvis samma
  uppfattning som du, och din uppfattning är kanske inte den bästa.
  Förstå att det finns andra med en annan uppfattning om ett visst tema.
  Var tolerant.
  Du har inte denna värld för dig själv.

\item[Anständighet:] aldrig får svordomar och oanständigheter yttras
  på banden.
  Ett sådant beteende säger ingenting om den person som de är avsedda för men
  mycket om den person som uttalar dem.
  Behåll ditt lugn i alla situationer.

\item[Förståelse:] Var snäll och förstå att alla inte är så smarta,
  så professionella eller så mycket expert som du.
  Om du vill göra något åt detta agera positivt (hur kan jag hjälpa till,
  hur kan jag förbättra, hur kan jag lära ut) i stället för negativt
  (med svordomar, förolämpningar etc.).
\end{description}

\subsection{Risken för konflikter}
\textbf{Endast en spelplan, etern}: alla radioamatörer vill spela sitt spel
eller utöva sin sport men det måste göras på en enda spelplan: våra amatörband.
Hundratusentals spelare på en enda spelplan leder ibland till konflikter.

Ett exempel: Plötsligt hör du någon ropa CQ på din frekvens (den frekvens du
har kört på en stund).
Hur är detta möjligt?
Du har varit igång här mer än en halvtimme på en helt ren frekvens!
Jo, visst är det möjligt; den där andra stationen tror kanske också att du stör
honom på HANS frekvens.
Kanske har skippet eller konditionerna ändrats?

\subsection{Hur undvika konflikter?}
\begin{itemize}
\item Genom att förklara för alla spelare vilka regler som gäller och genom
  att motivera dem att tillämpa dessa regler.
  De flesta konflikter orsakas av \textbf{okunskap}:
  många spelare känner inte till reglerna tillräckligt väl.

\item Dessutom hanteras många konflikter dåligt återigen på grund av
  \textbf{okunskap}.

\item Den IARU-etikhandbok som finns översatt på SSA:s webbplats avser att
  åtgärda denna brist på kunnande i huvudsak genom att lära ut hur man kan
  undvika konflikter av alla slag.
\end{itemize}

\subsection{Moraliska aspekter}
\begin{itemize}
\item I de flesta länder bryr sig myndigheterna inte om i detalj hur
  amatörerna uppför sig på banden, förutsatt att de håller sig till reglerna
  som myndigheten fastslagit.
\item Radioamatörerna anses vara \textbf{självstyrande}, detta betyder att
  självdisciplin måste utgöra basen i vårt agerande. Det betyder emellertid
  inte att radioamatörerna har en egen polisiär funktion!
\end{itemize}

\subsection{Förhållningsregler}

Vad menar vi med \textbf{förhållningsregler} (code of conduct)?
De är en uppsättning regler baserade på såväl \textbf{etiska} principer som
\textbf{trafikmässiga hänsyn}.

\begin{description}
\item[Etik] Etik bestämmer vår attityd och vårt allmänna uppförande
  som radioamatörer.
  Etik har med moral att göra.
  Etik utgör principerna för moral.

  Exempel: etiken säger oss att aldrig medvetet störa andra stationers
  radiotrafik.
  Detta är en moralisk regel.
  Det är omoraliskt att inte följa denna regel, likvärdigt med att fuska i en
  tävling.
\item[Praktiska regler] för att hantera alla olika aspekter av
  vårt uppförande behövs utöver etik också en uppsättning regler baserade på
  \textbf{trafikmässiga hänsyn} och på \textbf{praxis och sedvänja}.
  För att undvika konflikter behöver vi också praktiska regler som styr
  vårt beteende på amatörbanden eftersom vårt huvudintresse är att köra
  radio på de olika banden.
  Vi avser här mycket \textbf{praktiska regler} och \textbf{riktlinjer} för
  situationer som ej är etikrelaterade.
  De flesta trafikmetoder (hur genomföra ett QSO, var får man köra,
  vad betyder QRZ, hur använda Q-koderna) hör hit.
  Respekt för dessa trafikmetoder säkerställer optimalt resultat och
  effektivitet i våra QSO och kommer att vara nyckeln till att undvika
  konflikter.
  Dessa trafikmetoder har tillkommit som ett resultat av daglig radiotrafik
  under många år och som ett resultat av den pågående tekniska utvecklingen.
\end{description}
