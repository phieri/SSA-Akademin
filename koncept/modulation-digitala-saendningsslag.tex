\section{Digitala sändningsslag}

Här ges exempel på digitala modulationstekniker för kortvågstillämpningar inom
amatörradio.

De flesta digitala sändningsslagen för kortvåg är smalbandiga och bandbredden
kan i vissa fall endast vara några hertz.

Signalbehandlingen sker i den dator som programmet körs på och där datorns
in- och utgång för dess ljudkort är kopplade till amatöradioutrustningen.
Oftast är programmets styrning av sändning och mottagning också kopplad till
lämplig serieport, till exempel via dess USB-anslutning.

\subsection{RTTY}
\index{RTTY}
\index{FSK}
\index{Frequency Shift Keying (FSK)}
\index{Audio Frequency Shift Keying (AFSK)}
\index{AFSK}

\begin{historiabox}
Ett av de första digitala trafiksätten som användes av radioamatörer var
\emph{RTTY}, uttytt ''RadioTeleTYpe'', där man använde sig av så kallade
teleprintrar, automatiska skrivmaskiner som skrev ut text.

Emile Baudot konstruerade år~1874 ett system baserat på fem bitar,
som fortfarande används idag.
I augusti 1922 testade The US Department of Navy ''skriven telegrafi'' mellan
ett flygplan och en markstation.
Amerikanska kommersiella RTTY-system fanns aktiva så tidigt som 1932.
Under 50-talet började surplusutrustning komma ut på den amerikanska
marknaden och radioamatörerna var inte sena att prova den nya tekniken på
kortvågsbanden.
De kommersiella systemen körde med 50~baud, 75~baud eller 100~baud.

Amatörerna i USA körde med 45,45~baud, vilket motsvarar 300 tecken per minut.
De europeiska utrustningarna, bland annat framtagna av Siemens, arbetade med
50~baud men gick att justera ner till 45,45~baud.
45~baud är idag den vedertagna standarden över världen.
\end{historiabox}

RTTY använder FSK-modulering.
För att åstadkomma detta behöver man styra frekvensen så att den hoppar mellan
två frekvenser med en skillnad, ett så kallat ''skift'', på \qty{170}{\hertz}.

Äldre sändare behövde modifieras för att åstadkomma detta frekvensskift, men
med en SSB-sändare kunde man istället mata sändaren med två toner, som gav
samma resultat -- så kallad Audio Frequency Shift Keying (AFSK).

Med nyare amatörradiotransceivrar blev det senare den vanligaste förekommande
metoden att modulera sändaren.
Det innebar att man slapp modifiera utrustningen.

Idag kör de flesta radioamatörer RTTY med en dator och använder sig ofta av
AFSK-tekniken med hjälp av programvaror, med samma uppkoppling som man använder
för andra digitala trafiksätt.

\subsection{SSTV}
\index{SSTV}

\emph{Slow Scan Television (SSTV)} är en blandning av analog och digital teknik.
En SSTV sändning sker långsamt jämfört med traditionell TV, men är i grunden
rätt lik.
Varje linje sänds en efter en, men modulerad så att den kan sändas över en
SSB-radiolänk.
Intensiteten för varje pixel anger tonhöjden som moduleras, vilket därmed
innebär en FM-modulation.
Denna FM-modulerade ton skickas sedan över SSB.
I början på varje linje skickas ett 7-bitars tal med jämn paritet som anger
vilken modulationsform som används.
De olika modulationsformerna kan sedan hantera olika upplösningar samt variera
med avseende på svart-vitt eller färg.

\subsection{APRS}
\index{APRS}
\index{AX.25}
\index{TNC}
\label{modulation_aprs}

\emph{Automatic Packet Reporting System (APRS)} är en teknik för att
huvudsakligen över VHF och UHF förmedla GPS-position, väderdata, enkla
meddelanden och annat.
Den bygger på en teknik som heter \emph{AX.25}, som är en amatörradiospecifik
version av telekomstandarden X.25.
AX.25 moduleras över 1200~baud Bell~202 AFSK teknik på vanlig talkanal.
Ofta används en Terminal Node Controller (TNC) som gränssnitt mellan dator och
radio.

\subsection{PSK31}
\index{PSK31}
\label{modulation_psk31}

\begin{historiabox}
Namnet beskriver modulationstypen och överföringshastigheten i baud.
Det första programmet utvecklades specifikt för windowsbaserade datorer med
ljudkort av den engelska radioamatören Peter Martinez, G3PLX, och
introducerades i amatörradiovärlden 1998.
\end{historiabox}

Modulationen som används i PSK31 utvecklades från en idé av den polske
radioamatören Pawel Jalocha, SP9VRC, som hade tagit fram en mjukvara
''SLOWBPSK'' för Motorolas EVM-radio, vilket var ett radiosystem för att
utvärdera olika modulationsformer.
Istället för att använda den gängse metoden med frekvensskift baserades
''SLOWBPSK'' på polaritetsskiftning av fasläget.
Ett bra utformat PSK-baserat system kan ge bättre resultat än FSK, och kan
arbeta med smalare bandbredd än FSK.
Överföringshastigheten 31~baud valdes för att passa en genomsnittlig
skrivhastighet hos den gemene amatören.

\subsection{WSJT-X}
\index{WSJT-X}
\index{FSK441}
\index{JT6M}
\index{JT65}
\index{JT9}
\index{FT8}
\index{FSK}
\index{Frequency Shift Keying (FSK)}
\index{meteorer}
\index{troposcatter}
\index{EME}
\index{månstuds}
\index{WSPR}
\index{8FSK}

WSJT-X är ett program som används inom amatörradiohobbyn för kommunikation med
svaga signaler.
De flesta av dessa sändningsslag (se nedan) är så smalbandiga, att de inte
upptar större bandbredd än några hertz.

WSJT-X utgör en plattform för ett flertal olika tillämpningar där olika
varianter av i huvudsak FSK-modulering används.

\begin{historiabox}
Programmet skrevs initialt av Joe Taylor, K1JT, men är nu ett
öppen-källkodsprogram och utvecklas av ett litet team.
Joe Taylor fick sin utbildning i astronomi vid Harvard University.
Han var sedan verksam inom området astrofysik vid Princeton University,
varifrån han pensionerades 2006.
Joe Taylor tilldelades Nobelpriset i fysik år~1993.
\end{historiabox}

FSK441 används för att utvärdera överföringar via radiovågsreflekterande skikt
av laddade joner, som uppkommer från de spår som meteorer lämnar efter sig.

JT6M introducerades år~2002 och är avsett för kommunikation via bland annat
meteorreflektioner på \qty{6}{\metre}-bandet.

JT65, utvecklat och släppt år~2003, används för kommunikation via troposfären,
så kallat troposcatter, men också för kommunikation via reflektion mot månen
så kallad EME-trafik.

WSPR är i huvudsak tänkt för vågutbredningstester inom kortvågsområdet.
WSPR står för Weak Signaling Propagation Reporter och uttalas ''Whisper''.
WSPR är ett sändningsslag som använder amatörradiostationen som en radiofyr, en
så kallad beacon.
Sändning och mottagning sker i tvåminuterspass och efter varje sändningspass
rapporterar de stationer som mottagit signalen in sitt resultat till en öppen
databas över internet.
WSPR använder låga effekter, det går att nå europeiska stationer med under
\qty{100}{\milli\watt}, och andra kontinenter med effekter under några watt,
även med modesta antenner.
WSPR släpptes i sin första version 2008.

JT9 används för kortvågstrafik och är snarlikt JT65, men använder sig av en
FSK-signal med nio toner.
JT9 använder sig av mindre än \qty{16}{\hertz} bandbredd.

FT8 utvecklades och släpptes år~2017 och använder sig av en 8FSK-signal.
FT8 är att föredra vid så kallat multi-hop via E-skikt, där signalerna utsätts
för fädning och där öppningarna mot andra stationer är korta så att man behöver
slutföra kommunikationen inom en kort tid.

\subsection{FreeDV}
\index{FreeDV}

FreeDV skiljer sig mot de sändningsslag som nämnts ovan genom att detta är tänkt
för digitalt tal på kortvåg.

FreeDV skapades av en grupp radioamatörer från olika länder som arbetade
med kodning, utformning, användargränssnitt och testning.
FreeDV släpptes år~2015.

FreeDV är tänkt att användas på kortvåg med SSB-modulerade radiostationer,
men kan också användas med AM- eller FM-modulering.
Fördelen ska vara att överföringen blir mer robust samt att signaleringen är
utformad för att motverka påverkan av fädning.

FreeDV använder en något mer komplex modulering.
Man använder sig av ett flertal bärvågor inom dess bandbredd på
\qty{1,25}{\kilo\hertz}.
Bärvägorna ligger med \qty{75}{\hertz} mellanrum och varje bärvåg moduleras med
varianter av PSK-modulering.
Bandbredden är hälften (\qty{1,25}{\kilo\hertz}) av en normal SSB-bandbredd
(\qty{2,4}{\kilo\hertz}).
