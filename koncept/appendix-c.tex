\chapter{Omräkning mellan dB och kvoten av tal}
\label{decibel_2}

\noindent
Benämningen Bel kommer från namnet på amerikanen Alexander Graham
Bell, som år 1876 uppfann den första praktiskt användbara telefonen
efter idéer från tysken Philipp Reiß.

Inom teletekniken används begreppet decibel för att beskriva förlopp
av effekt, ström och spänning. Begreppet förekommer även i andra
sammanhang, till exempel akustik där det istället är fråga om ljudtryck.

Måtten i det metriska systemet är alldagliga och ingen finner det
märkligt att det till exempel går tio decimeter på en meter. Däremot är
begreppet decibel ovant för många.

Räkning med decibel grundas på användning av logaritmer, som är ett
bekvämt sätt att uttrycka och behandla talvärden. Detta har i
korthet förklarats i avsnitt \ref{effect och energi}. Här beskrivs ett
omräkningsförfarande med hjälp av tabeller.

\begin{description}
\item[Decibel] är ett dimensionslöst uttryck för graden av dämpning
  alternativt förstärkning.

\item[Dämpning] är följden av att vissa komponenter bromsar elektrisk
  ström.

\item[Förstärkning] innebär att en aktiv komponent kan styra en större
  elektrisk ström och därmed större effekt än den själv styrs med.
\end{description}

\section{Decibel över 1 mW vid 50 ohm [dB(m)]}

Som nu beskrivits är uttrycket decibel ett logaritmiskt mått för hur
två effekter förhåller sig till varandra. När de jämförda effekterna
uppträder över lika stora impedanser, kan även förhållandet mellan två
spänningar eller två strömmar uttryckas i decibel.
I samtliga fall rör det sig om förhållandet mellan två storheter --
\emph{aldrig absoluta storheter}.

Exempel: Ett drivsteg i en sändare drivs med 1~watt och avger 10~watt.
Effektförhållandet är 10:1 och effektförstärkningen är 10 gånger eller
\SI{10}{\decibel}.
Slutförstärkaren i samma sändare drivs med 10~watt från drivsteget och avger
100~watt till antennen.
Även i detta fall är effektförhållandet 10:1 och effektförstärkningen 10 gånger
eller \SI{10}{\decibel}.

Slutförstärkaren hanterar en 10 gånger så hög effektnivå som drivsteget och ändå
är förstärkningen \SI{10}{\decibel} i båda fallen.
Decibel är m.a.o. dimensionslöst.

Men om en av två jämförda effekterna alltid är densamma och väl
definierad så medges nya möjligheter. Den effekt som ska
kvantifieras kan nu ställas i förhållande till den kända
referenseffekten. Med denna förutsättning kan även de absoluta
effektnivåerna, till exempel genom en sändare uttryckas i decibel. Detta
tillgår på följande sätt.

Det är mycket vanligt att in- och utgångarna i HF-utrustningar utförs
med en impedans av \SI{50}{\ohm}.
För god anpassning väljs då koaxialkablarna mellan apparaterna med en
karaktäristisk impedans av \SI{50}{\ohm}.

\emph{Det har utbildats en praxis, att referensvärdet vid jämförelse
  av signalnivåer i radiosystem ska vara en milliwatt (\SI{1}{\milli\watt})
  utvecklad i en belastning med impedansen \SI{50}{\ohm}.}

Signalnivåer över belastningen \SI{50}{\ohm} kan uttryckas i dB(m), där (m)
står för milliwatt, varvid referenseffekten \SI{1}{\milli\watt} är 0\,dB(m) vid
\SI{50}{\ohm}.

Det spänningsfall som bildas över belastningen \SI{50}{\ohm} vid effektnivån
0\,dB(m) är
%%
\[U = \sqrt{P\cdot R} = \sqrt{1\cdot 10^{-3} \cdot 50} \approx \SI{0.224}{\volt}\]
%%
Den ström som flyter genom belastningen \SI{50}{\ohm} vid effektnivån 0\,dB(m)
är
%%
\[
I = \sqrt{\frac{P}{R}} = \sqrt{\frac{1\cdot 10^{-3}}{50}} = \SI{0.0045}{\ampere} = \SI{4.5}{\milli\ampere}
\]
%%
Strömmen \SI{4,5}{\milli\ampere} genom belastningen \SI{50}{\ohm} motsvarar
således 0\,dB(m).

Varje annan effekt, spänningsfall och ström som uppstår vid en belastning av
\SI{50}{\ohm} kan jämföras med respektive referensvärden \SI{1}{\milli\watt},
\SI{0,22}{\volt} och \SI{4,5}{\milli\ampere}.
\emph{dB(m) är ett absolut och logaritmiskt mått.}

\noindent
Effekt:
%%
\begin{gather*}
  a [dB(m)] = 10 \log\frac{P_{[50\Omega]}}{1[mW_{50\Omega}]} \\
  P_{50} = 1 [mW] \cdot 10^{\frac{a}{10}}
\end{gather*}
%%
Ström:
%%
\begin{gather*}
  0 dB(m) = 4.47 mA_{50} \\
  a [dB(m)] = 20 \log\frac{I_{50}}{4.47}
\end{gather*}
%%
Spänning:
%%
\begin{gather*}
  0 dB(m) = 0.223 V_{50} \\
  a [dB(m)] = 20 \log\frac{U_{50}}{0.223} \\
  U_{50} = 0.223 \cdot 10^{\frac{a}{20}}
\end{gather*}

\section{Sambandet mellan spänning över 50 ohm och dB(m)}

\begin{center}
\begin{tabular}{S[table-format=-3]|lp{1cm}S[table-format=-2]|l}
  \multicolumn{1}{c|}{dB(m)} & V & &  \multicolumn{1}{c|}{dB(m)} & V \\
  \cline{1-2} \cline{4-5}
  -40 & 0,00224 & & & \\
  -30 & 0,00707 & & & \\
  -20 & 0,0224  & & & \\
  -10 & 0,0707  & & & \\
  0   & 0,224   & & & \\
  1   & 0,251   & & 11 & 0,793 \\
  2   & 0,282   & & 12 & 0,890 \\
  3   & 0,316   & & 13 & 0,999 \\
  4   & 0,354   & & 14 & 1,121 \\
  5   & 0,398   & & 15 & 1,257 \\
  6   & 0,446   & & 16 & 1,411 \\
  7   & 0,501   & & 17 & 1,583 \\
  8   & 0,562   & & 18 & 1,776 \\
  9   & 0,630   & & 19 & 1,993 \\
  10  & 0,707   & & 20 & 2,236 \\
\end{tabular}
\end{center}

\emph{dB(W) är ett annat absolut mått.}
Effektnivåer över en belastning kan också uttryckas i dB(W), där (W) står
för watt.
Referenseffekten är då \SI{1}{\watt}, det vill säga 0\,dB(W).
Liksom med dB(m) anges impedansen i den belastning, som effekten utvecklas
över.
Exempelvis motsvarar 26\,dB(W) \SI{398}{\watt} (se tabellen för sambandet mellan
effektförhållande och \si{\decibel}).
