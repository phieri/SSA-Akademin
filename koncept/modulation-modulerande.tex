\newpage
\section{Modulerande signaler}
\harecsection{\harec{a}{1.7.1}{1.7.1}}
\index{modulerande signaler}

\subsection{Basband}
\index{basband}

Basband är ett frekvensområde för en modulerande signal.
Det finns ett basband för alla slags modulerande signaler, vare sig de är
analoga eller digitala.
Det kan finnas mer än ett basband i en komplett modulationsprocess.
Till exempel är en nycklad ton, som går till sändaren genom mikrofoningången,
dess analoga basband medan nycklingspulserna till tongeneratorn är dess
digitala basband.

Bild~\ssaref{fig:BildII1-23} illustrerar modulerade signaler.
Ett vanligt sätt att överföra information över radio är med telefoni, det vill
säga tal.

Frekvensområdet \SIrange{300}{3000}{\hertz} räcker för god förståelighet av tal.
Dels är örat känsligast inom det området och dels finns där den mesta energin
i talet.

Mikrofonen tar upp de lufttrycksvariationer som uppstår när man talar och
omvandlar dem till elektriska svängningar.
Svängningarna varierar mellan positiva och negativa spänningsvärden.

\bigskip

\textbf{Försök}

\begin{enumerate}
\item Anslut en mikrofon till ett oscilloskop och studera spänningsförloppen
  för olika slags ljud, toner, tal osv. som funktion av tiden.
  På bilden är dessa svängningar mycket förenklade, till exempel sinusformade.

\item Anslut en högtalare och ett oscilloskop till en LF-generator, vars
frekvens och amplitud kan ändras. Lyssna på ljud med låg och hög frekvens samt
på svaga och starka ljud.
En baston har låg frekvens och en diskantton har hög frekvens.
En svag ton har liten amplitud och en stark ton har stor amplitud.
\end{enumerate}
