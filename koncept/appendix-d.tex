\chapter{S-enheter och dB}
\label{s-enhet}

%% k7per: Make table double wide and float to bottom of page, dependengt on stfloats, using table*.
%% TODO: Needs label.
\begin{table*}[b]
  \vspace{10ex}
	\centering
	\begin{tabular}{l|SSS|SSS}
    S-Meter  & \multicolumn{3}{c|}{Under \SI{30}{\mega\hertz}} & \multicolumn{3}{c}{Över \SI{30}{\mega\hertz}} \\
    värde    & \multicolumn{1}{c}{dBm} & \multicolumn{1}{c}{(\si{\micro V} vid \SI{50}{\ohm})} & \multicolumn{1}{c|}{\si{dB\micro V}} & \multicolumn{1}{c}{dBm} & \multicolumn{1}{c}{(\si{\micro V} vid \SI{50}{\ohm})} & \multicolumn{1}{c}{\si{dB\micro V}} \\
    \hline
    S9 +\SI{40}{\decibel} & -33  & 5000  & 74        & -53  & 500  & 54  \\
    S9 +\SI{30}{\decibel} & -43  & 1600  & 64        & -63  & 160  & 44  \\
    S9 +\SI{20}{\decibel} & -53  & 500   & 54        & -73  & 50   & 34  \\
    S9 +\SI{10}{\decibel} & -63  & 160   & 44        & -83  & 16   & 24  \\
    S9                    & -73  & 50    & 34        & -93  & 5    & 14  \\
    S8                    & -79  & 25    & 28        & -99  & 2,5  & 8   \\
    S7                    & -85  & 12,6  & 22        & -105 & 1,26 & 2  \\
    S6                    & -91  & 6,3   & 16        & -111 & 0,63 & -4  \\
    S5                    & -97  & 3,2   & 10        & -117 & 0,32 & -10 \\
    S4                    & -103 & 1,6   & 4         & -123 & 0,16 & -16 \\
    S3                    & -109 & 0,8   & -2        & -129 & 0,08 & -22 \\
    S2                    & -115 & 0,4   & -8        & -135 & 0,04 & -28 \\
    S1                    & -121 & 0,21  & -14       & -141 & 0,02 & -34 \\
  \end{tabular}
  \caption{Tabell över S-värden, spännigar och effekter}
\end{table*}

\noindent
I kommunikationsradiomottagare brukar det nästan alltid finnas en anordning som
mäter och visar styrkan av mottagna signaler.

Eftersom spänningen från antennen in i mottagaren kan variera över ett stort
område, är det praktiskt att uttrycka styrkevärdena med en logaritmisk
måttenhet, så kallad S-enhet.

Signalspänningen mäts över en impedans av \SI{50}{\ohm}.

Eftersom S-enheten är logaritmisk, så motsvarar till exempel signalstyrkan S8
halva signalspänningen, det vill säga \SI{25}{\micro\volt} eller
\SI{-6}{\decibel} jämfört med S9.
Om halveringen fortsätts, fås att S0 (noll) motsvarar en signalstyrka av
\SI{0,1}{\micro\volt}.

I en kortvågsmottagare alstras det ett internt brus med en nivå av
åtminstone \SI{0,1}{\micro\volt}.
Detta brus blandas med den inkommande signalen.
En insignal med en styrka under brusnivån (S0) kommer alltså inte att
kunna höras.
Vid högre signalstyrkor än S9 anges styrkan som S9 +ett antal dB.
Det är då frågan om mycket starka signaler.

Följande tabell gäller för det ideala sambandet mellan S-enheter och
signalstyrkor över två alternativa brusnivåer.

Signalstyrkan mäts vid mottagarens antenningång, varför skillnaden i
signalstyrkan olika antenner och mottagningsriktningar samt dämpningen
i antenn och nedledning kan behöva bedömas.

I kortvågsområdet (under \SI{30}{\mega\hertz}) uppträder ett atmosfäriskt
bredbandigt brus tillsammans med bruset från den stora mängden rundradio- m.fl.
andra starka sändare.
Detta brus är mer dominerande än mottagarens interna brus.
I praktiken har de flesta KV-mottagare en högre brusnivå än
\SI{0,1}{\micro\volt}.

Över \SI{30}{\mega\hertz} däremot, är det mest mottagarens interna brus som
sätter gränsen för hörbarheten av svaga signaler.
Med samma S-skala som för kortvågsområdet, börjar man uppfatta signaler i
bruset utan att S-metern ger utslag.

Vid IARU Region~1-konferensen 1978 i Miskolcz föreslog de nationella
föreningarna VERON (Nederländerna) och RSGB (Storbritannien) en annan
S-skala över \SI{30}{\mega\hertz}.
Vid konferensen 1981 i Brighton antogs förslaget som rekommendation.

Mätningar ska i båda fallen göras med en kvasi-toppvärdesdetektor med en stigtid
av \SI{10}{\milli\second} \(\pm\)\SI{0,2}{\milli\second} och en falltid av
\SI{500}{\milli\second}.













