\chapter{Elsäkerhet}
\label{Elsäkerhet}
\index{Elsäkerhet}

\section{Människokroppen}
\harecsection{\harec{a}{10.1}{10.1}}

\subsection{Elektrisk chock}

Människokroppen är ett komplicerat elektrokemiskt system, som främst
kontrolleras av hjärnan.
Musklerna styrs av svaga elektriska strömimpulser genom nervsystemet.
Främmande strömmar genom kroppen kan störa kroppsfunktioner och kan i olyckliga
fall göra stor skada.
Styrkan och frekvensen på strömmarna avgör skadans art och omfattning.

\paragraph{Elektrisk chock kan döda av flera orsaker.}
En orsak är att hjärtrytmen störs.
Hjärtkammarflimmer och hjärtstillestånd kan lätt uppstå.
Flimmer innebär att hjärtat arbetar okontrollerat och med kraftigt nedsatt
eller helt upphävd pumpfunktion.
Hjärtstillestånd inträffar lätt av hög spänning.
Av otillräcklig blodtillförsel blir det syrebrist i hjärncellerna, som då
skadas snabbt.
Medvetslöshet inträder redan efter ett fåtal sekunder.

En annan orsak är andningsstillestånd genom att andningscentrum blockeras.
Det kan hända när strömmen från en högspänningskondensator går genom kroppen.

\subsection{Hjärt- och lungräddning, HLR}
\index{hjärt- och lungräddning}
\index{HLR|see {hjärt- och lungräddning}}

Vid hjärtstillestånd, hjärtkammarflimmer eller andningsstillestånd ska
hjärt- och lungräddning påbörjas omedelbart då obotliga hjärnskador av
syrebrist kan uppstå inom några få minuter.
Finns en hjärtstartare, AED, i närheten bör den användas så skyndsamt som
möjligt.

\textbf{Glöm inte att ringa efter hjälp! Ring 112!}

Broschyren \emph{Vägledning vid elskada} kan laddas ner eller beställas från
Elsäkerhetsverkets webbplats
<\href{https://www.elsakerhetsverket.se}{\texttt{www.elsakerhetsverket.se}}>.

Vårdguiden 1177 <\href{https://www.1177.se}{\texttt{www.1177.se}}> har
instruktioner för hjärt- och lungräddning (HLR).

Svenska rådet för Hjärt- och Lungräddning \\
<\href{https://www.hlr.nu}{\texttt{www.hlr.nu}}> har beskrivningar och
instruktionsfilmer för hjärt- och lungräddning.

\subsection{Resistansen genom människo\-kroppen}

Vid kontakt med ett strömförande föremål kommer kroppen att bli en del
av strömkretsen. Det flyter då en främmande ström genom kroppen.

Strömstyrkan följer Ohms lag och beror av strömkällans spänning och inre
resistans samt av övergångsresistansen i huden och kroppens inre resistans.

Övergångsresistansen minskar med fuktigare hud samt med större kontaktyta och
större kontakttryck. Beröringsspänningen inverkar också.
Vid spänningar över cirka \SI{75}{\volt} minskar övergångsresistansen med ökande
spänning.
Vid allvarliga brännskador minskar övergångsresistansen särskilt mycket.
Den totala resistansen genom kroppen blir då nära lika med dess inre resistans
-- ungefär \SI{500}{\ohm}.

\begin{center}
\begin{minipage}{0.19\columnwidth}
\Huge{\fontencoding{U}\fontfamily{futs}\selectfont\char 66\relax}
\end{minipage}
\begin{minipage}{0.7\columnwidth}
  Experimentera inte med detta! Det kan vara livsfarligt.
\end{minipage}
\end{center}


\subsection{Strömmens inverkan på människan}

Sjukvården skiljer på verkan av strömstöt, strömgenomgång och ljusbåge.

En strömstöt kan tyckas ofarlig men kan leda till okontrollerade rörelser,
fallskada eller beröring av andra spänningsförande föremål.

Vid en strömgenomgång utjämnas en elektrisk potentialskillnad genom kroppen
vilket utöver hjärtstillestånd, hjärtkammarflimmer, och andningsstillestånd
kan leda till blodpropp, muskelskador, njurskador eller inre brännskador.

Vid en ljusbågsolycka ökar risken för kraftiga brännskador på grund av den
höga temperaturen i ljusbågen.
En ljusbåge kan även orsaka skador på ögonen på grund av bländning eller den
stora mängden UV-ljus.

%% \begin{center}
%% \begin{minipage}{0.19\columnwidth}
%% \Huge{\fontencoding{U}\fontfamily{futs}\selectfont\char 66\relax}
%% \end{minipage}
%% \begin{minipage}{0.7\columnwidth}
%% Personer som drabbats av olycka med
%% \end{minipage}
%% \end{center}
%% \begin{itemize}
%% \item högspänning
%% \item lågspänning med strömgenomgång genom bålen
%% \item som är omtöcknade eller medvetslösa efter strömolycka
%% \item som har drabbats av brännskada
%% \item som visar tecken på nervskada till exempel förlamning
%% \end{itemize}
%% \textbf{ska omedelbart till sjukhus för akut behandling.}

\bigskip
\noindent
\begin{minipage}{0.19\columnwidth}
\Huge{\fontencoding{U}\fontfamily{futs}\selectfont\char 66\relax}
\end{minipage}
\begin{minipage}{0.7\columnwidth}
Personer som drabbats av olycka med
\begin{itemize}
\item högspänning
\item lågspänning med strömgenomgång genom bålen
\item som är omtöcknade eller medvetslösa efter strömolycka
\item som har drabbats av brännskada
\item som visar tecken på nervskada till exempel förlamning
\end{itemize}
\textbf{ska omedelbart till sjukhus för akut behandling.}
\end{minipage}

\vspace{1ex}
%Starka strömmar ger häftiga muskelkramper och/eller brännskador.
Häftiga muskelkramper och/eller brännskador kan uppkomma av starka strömmar.
Muskelkramp kan förekomma redan vid strömmar under \SI{10}{\milli\ampere}.
För vuxna, friska människor är det direkt farligt när strömmen överstiger
detta värde.
För unga eller sjuka kan strömmar under \SI{10}{\milli\ampere} vara direkt
farliga.

Strömstyrkan påverkar kroppen olika från fall till fall och det är osäkert
vilken strömstyrka som är farlig.
Det finns både de som överlevt höga strömmar och de som inte har klarat någon
milliampere.
Strömmar som går genom hjärta eller hjärna är särskilt farliga.
När man arbetar med elektriska apparater under spänning, bör man för säkerhets
skull hålla den ena handen i fickan!

\subsection{Påverkan av elektromagnetiska fält}

Undersökningar har visat att vistelse i starka elektromagnetiska fält
kan kan påverka människan.
Personer som har varit utsatta för kraftig exponering av fält har bland annat
klagat över svettningar och huvudvärk.
Det forskas omkring dessa fenomen.

Elektromagnetiska fält kan förorsaka fel i elektronikutrustningar.
Halvledare är särskilt känsliga för kraftfält.
Det är möjligt att känsliga instrument, hjärtstimulatorer (pacemaker) etc. kan
påverkas av högfrekventa elektromagnetiska fält från radiosändare.
När du använder en sändare, mobiltelefon etc. och någon får svårigheter med
hjärta eller andning så ska du omedelbart stänga av din apparat helt!
Med tiden utvecklas störningsokänsligare elektronik, men säker mot störningar
kan man aldrig vara. Se vidare i kapitel \ssaref{EMF}.

\subsection{Normer för fältstyrkor}

Det finns flera olika normer och rekommendationer för elektromagnetiska
fältstyrkor. Några av dessa normer har till exempel syftet att olika slags
apparater ska kunna samexistera och därför fungera utan att påverkas av
elektromagnetiska fält eller stråla ut elektromagnetiska fält överstigande
givna gränsvärden (EMC).

Andra normer och råd har till syftet att skydda arbetstagare eller individer
ur allmänheten från akuta biologiska effekter när de exponeras för
elektromagnetiska fält.

Strålsäkerhetsmyndigheten har genom utgivandet av SSMFS 2008:18 publicerat
allmänna råd om begränsning av allmänhetens exponering för elektromagnetiska
fält.
Dessa råd bygger på rekommendationer från Europeiska unionens råd.
Se vidare i kapitel \ssaref{EMF}.
