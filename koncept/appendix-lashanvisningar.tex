\onecolumn

\chapter{Läsanvisning för certifikatprov}

\section{Teknikdelens läsanvisningar}


\begin{table}[H]
	\small
\begin{tabular}{rll}
\textbf{Nr} & \textbf{Innehåll} & \textbf{Avsnitt}\\ \hline\hline
T1 & ledare, halvledare och isolatorer & 
\ssaref{konduktivitet}, \ref{ledare}, \ref{isolator}, \ref{halvledare}\\ \hline
T2 & potential, spänningsfall, resistans, ohms lag &
\ssaref{spänning}, \ref{spänning.symboler}, \ref{elektrisk_ström}, \ref{strömkrets}, \ref{resistans},\ref{ohms_lag}\\ \hline
T3 & elektrisk effekt, joules lag &
\ssaref{elektrisk_effekt}, \ref{joules_lag}\\ \hline
T4 & batterier, och batterikapacitet & 
\ssaref{batterikapacitet}\\ \hline
T5 & inre resistans, kortslutningsström & 
\ssaref{inre_resistans}, \ref{kortslutningsström}\\ \hline
T6 & serie- och parallellkopplade kraftkällor &
hela \ssaref{kraftkällor_serie_parallell}\\ \hline
T7 & elektriska fält och fältstyrka &
\ssaref{elektrisk_fälststyrka}, \ref{elektrostatik skärmning}\\ \hline
T8 & magnetiska fält och fältstyrka &
\ssaref{magfält_ström}, \ref{magnetisk_fältstyrka}\\ \hline
T9 & våglängd och frekvens &
\ssaref{utbredningsmodeller}, \ref{elektromagnetiska_fält}\\ \hline
T10 & toppvärde, amplitud, effektivvärde &
\ssaref{toppvärde}, \ref{peak-to-peak-värde}, \ref{effektivvärde}\\ \hline
T11 & periodtid och frekvens&
\ssaref{period}, \ref{frekvens}\\ \hline
T12 & övertoner &
\ssaref{övertoner}\\ \hline
T13 & analog modulation och analoga sändningsslag&
\ssaref{modulationssystem}, \ref{sändningsslag}, \ref{kännetecken_modulerade_signaler}, 
\ssaref{bandbredd_modulation},  \ref{modulation_beskrivningskod}\\
 && \ssaref{modulation_am}, \ref{modulation_cw}, \ref{modulation_ssb}, 
 \ssaref{modulation_vinkel}, \ref{modulation_fm}\\ \hline
T14 & digital modulation och digitala sändningsslag &
hela \ssaref{modulation_digital}, \ref{bitfel_detektion}, \ref{modulation_aprs}, 
\ssaref{modulation_psk31}\\ \hline
T15 & decibel, dBm, verkningsgrad &
\ssaref{effekt_db}, \ref{dBm}, \ref{verkningsgrad}\\ \hline
T16 & digital signalbehandling, sampling, kvantisering & \\
   & samplingsfrekvens, D/A och A/D-omvandlare &
\ssaref{digital_signalbehandling}, \ref{sampling}, \ref{nyquist}, \ref{ADC-DAC}\\ \hline
T17 & resistorer (motstånd) & 
\ssaref{enheten_ohm}, \ref{fasta_resistorer_linjära}, \ref{fasta_resistorer_olinjära}\\
& kondensatorer & 
\ssaref{resistor_temperaturkoefficient}, \ref{kondensator_allmänt}--\ref{kapacitiv_reaktans}\\ 
& induktorer (spolar) &
\ssaref{induktor_allmänt}, \ref{enheten_henry}--\ref{induktiv_reaktans} \\
& tranformatorer & 
\ssaref{ideal_transformator} \\ \hline
T18 & dioder och diodtillämpningar &
\ssaref{dioden_allmänt}, \ref{diod_zener}, \ref{diod_led}\\ \hline
T19 & transistorer och förstärkningsfaktor, strömställare &
\ssaref{transistor_allmänt}, \ref{transistor_förstärkningsfaktor}, \ref{transistor_pnp}, 
\ssaref{transistor_strömställare} \\ \hline
T20 & Operationsförstärkare (jmfr buffertsteg) & 
\ssaref{op-amp} \\ \hline
T21 & serie- och parallellkopplade resistorer &
\ssaref{seriekopplade_resistorer}, \ref{parallellkopplade_resistorer}\\ \hline
T22 & spänningsdelare & 
\ssaref{spänningsdelare}\\ \hline
T23 & serie- och parallellkopplade kondensatorer & 
\ssaref{parallellkopplade kondensatorer}, \ref{seriekopplade_kondensatorer} \\ \hline
T24 & galvaniskt kopplade induktorer & 
\ssaref{galvaniskt_kopplade_induktorer}, \ref{induktor_urkoppling}\\ \hline
T25 & impedans och ohms lag vid växelström & 
\ssaref{impedans}, \ref{ohms_lag_växelström}, \ref{impedans_resonant_krets}\\ \hline
T26 & filter & 
\ssaref{filter} (inl.), \ref{lågpassfilter}, \ref{bandfilter_kristall} \\ \hline
T27 & kraftförsörjning och likriktare &
\ssaref{kraftförsörjning} (inl.), \ref{likriktning}, \ref{glättningskretsar} (inl.), \ref{spänningsstabilisering}\\ \hline
T28 & förstärkarsteg & 
\ssaref{förstärkarsteg_allmänt}, \ref{förstärkare_grundkoppling}, 
\ssaref{förstärkare_utstyrningskontroll}\\ \hline
T29 & detektorer och demodulatorer & 
\ssaref{detektorer_allmänt}, \ref{fm_detektor} (inl.)\\ \hline
T30 & svängningar o oscillator, kristallosc., PLL, buffert & 
\ssaref{svängningar_alstring}--\ref{svängningar_LC-oscillator}, 
\ssaref{kristalloscillator}, \ref{PLL}, \ref{buffertsteg}\\ \hline
T31 & obalans i antennsystem & 
\ssaref{obalans_antennsystem}\\ \hline
T32 & mottagare, raka, superheterodyn, AGC & 
\ssaref{mottagare_bättre_hf}, \ref{selektion_direktblandade}, \ref{passband_spegelfrekvens}, 
hela \ssaref{superheterodynmottagaren}, \ref{AGC} (inl.)\\ \hline
T33 & brusspärr och selektivitet & 
\ssaref{brusspärr}, \ref{tonöppning}, \ref{subton}\\ \hline
T34 & selektivitet, spegelfrekvenser, mottagarkänslighet & 
\ssaref{selektivitet}, \ref{spegelfrekvens_mottagare} (inl.), 
\ssaref{bandbredd_fm}, \ref{signalkänslighet_brus}\\ \hline 
T35 & sändare, blockscheman, egenskaper, duplex & 
\ssaref{sändare_blockschema}-\ref{sändare_frekvensblandning}, \ref{utgångsimpedans}, \ref{cw-klickar}, \ref{splatter}, \ref{duplex}\\ \hline
T36 & antenner och antennvinst, polarisation &
\ssaref{antenner_allmänt}--\ref{antenner_impedans}, 
\ssaref{antenner_ståendevåg}--\ref{antenner_antennvins}, 
\ssaref{polarisation_hf}, \ref{polarisation_vhf}\\ \hline
T37 & antenner (kortvåg, VHF, UHF och SHF) &
\ssaref{ändmatad_halvvågsantenn}, \ref{jordplanantenn}, 
\ssaref{antenner_vhf_allmänt}, \ref{antenner_vhf_yagi}\\ \hline
T38 & transmissionsledningar & 
\ssaref{avstämd_matarledning}, \ref{oavstämd_matarledning}, \ref{stående_vågor}-
\ssaref{antenner_balansering}, Tabell \ref{Kabeldämpning}\\ \hline
T39 & vågutbredning och jonosfären & 
\ssaref{vågutbredning_reflektion}, \ref{vågutbredning_jonosfärskikten}--
\ssaref{d-skiktet}, \ref{e-skiktet}--\ref{sporadiskt_e}, hela 
\ssaref{solens_inverkan_jonosfären}\\ \hline
T40 & vågutbredning på kortvåg &
\ssaref{markvåg}-\ref{rymdvåg}, \ref{fädning}--\ref{om_kortvågsbanden}, hela \ref{vågutbredning_vhf}\\ \hline
T41 & mätning av likström, ståendevåg & 
\ssaref{mäta_likspänning}, \ref{mäta_ståendevåg}\\ \hline
T42 & konstlast, fältstyrkemätare & 
\ssaref{konstlast}, \ref{fältstyrkemätare}\\ \hline 
\end{tabular}
\normalsize
\end{table}

\newpage

\section{Reglementesdelens läsanvisningar}

\begin{table}[H]
\small
\begin{tabular}{rll}
\textbf{Nr} & \textbf{Innehåll} & \textbf{Avsnitt}\\ \hline\hline
R1 & fonetiska alfabeten & 
hela \ssaref{fonetiska_alfabet}, tabellen \ref{tab:bokstavering-svenska}\\ \hline
R2 & Q-koden &
hela \ssaref{q-koden}, tabellen \ref{tab:q-kod}\\ \hline
R3 & trafikförkortningar & 
hela \ssaref{trafikförkortningar}, tabellen \ref{tab:trafikforkortningar}\\ \hline
R4 & nödsignaler & 
hela \ssaref{nödsignaler} utom \ref{nödfrekvens}\\ \hline
R5 & anropssignaler & 
\ssaref{anropssignaler}--\ref{cq dx och split}, \ref{innehåll i förbindelse}--\ref{kryptering av radiomeddelande} \\ \hline
R6 & bandplaner &
hela \ssaref{bandplaner}, appendix \ref{bandplaner2} HF--UHF (veta gränser för CW/SSB)\\ \hline
R7 & bestämmelser, radioreglementen & 
\ssaref{ITU radioreglemente}, \ref{amatörradio definitioner}, \ref{regioner}\\ \hline
R8 & CEPT &
hela \ssaref{CEPT} \\ \hline
R9 & svensk lag och föreskrift & 
hela \ssaref{svensk lag och föreskrift} \\ \hline
\end{tabular}
\normalsize
\end{table}

\section{Säkerhet läsanvisningar}

\begin{table}[H]
	\small
	\begin{tabular}{rll}
		\textbf{Nr} & \textbf{Innehåll} & \textbf{Avsnitt}\\ \hline\hline
		S1 & EMC elektromagnetisk kompatibilitet & 
		hela kapitel  \ssaref{EMC}\\ \hline
		S2 & EMF elektromagnetiska fält &
		hela kapitel  \ssaref{EMF}\\ \hline
		S3 & Elsäkerhet & 
		hela kapitel  \ssaref{Elsäkerhet}\\ \hline
	\end{tabular}
	\normalsize
\end{table}

\twocolumn
