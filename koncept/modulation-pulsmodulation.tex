\newpage
\section{Pulsmodulation}
\index{pulsmodulation}
\index{PWM}
\index{PAM}
\index{PPM}

Pulsmodulation används mest i mikrovågsområdet.
Pulsmodulerade signaler sänds vanligen som en serie korta pulser åtskilda av
relativt långa pauser utan modulering.

En typisk sändning kan bestå av pulser med en längd av \qty{1}{\micro\second}
och en frekvens av \qty{1000}{\hertz}.
Toppeffekten på en pulssändning är därför mycket högre än dess medeleffekt.

Före WARC~79 var symbolen för all pulssändning P.
Därefter används P endast för omodulerade pulståg.
Annan pulsmodulation har följande symboler

\begin{description}
\item[K] -- puls-/amplitudmodulation (PAM)
\item[L] -- pulsviddmodulation (PWM)
\item[M] -- pulsposition/fasmodulation (PPM)
\item[Q] -- vinkelmodulation under pulsen
\item[V] -- kombination av dessa eller annat sätt.
\end{description}

\begin{table*}[ht]
\begin{center}
\begin{tabular}{|L{.12\textwidth}|L{.18\textwidth}|L{.18\textwidth}|L{.18\textwidth}|L{.19\textwidth}|}
\hline
  Sändningsslag &
    Amplituden på LF-signalen &
    Tonhöjden på LF- signalen påverkar & 
    Bandbredden b förhåller sig till &
    För stor amplitud på LF-signalen medför \\
  \hline % =======================================================
  A3E (AM) & 
    amplituden i båda sidbanden &
    sidofrekvensernas avstånd från bärvågen &
    LF-signalens högsta frekvens & 
    övermodulering och för stor bandbredd \\
\hline
  J3E (SSB) & 
    amplituden på utsänt sidband & 
    sidofrekvensernas avstånd från bärvågen  & 
    skillnaden mellan LF-signalens högsta och lägsta frekvens & 
    för stor bandbredd, överstyrning av förstärkarsteg \\
\hline
  F3E (FM) &
    deviationen & 
    hastigheten på bärvågens frekvensändring &
    dubbla summan av största deviation och högsta LF-frekvens & 
    för stor deviation, för stor bandbredd \\
  \hline % =======================================================
\end{tabular}
\end{center}
\caption{Jämförelse mellan några vanliga sändningsslag inom amatörradio}
\end{table*}
