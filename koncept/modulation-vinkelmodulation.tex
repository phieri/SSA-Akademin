\newpage
\section{Vinkelmodulation}
\harecsection{\harec{a}{1.8.3a}{1.8.3a}}
\index{vinkelmodulation}
\label{modulation_vinkel}

Termen vinkelmodulation är samlingsnamnet för frekvensmodulation (FM) och
fasmodulation (PM).
Ofta sägs utrustningar vara för frekvensmodulation när de antingen är för
frekvens- eller fasmodulation.
Det finns alltså skillnader och likheter mellan dessa system, vilka emellertid
inte är oberoende av varandra, eftersom frekvensen i en signal inte kan
varieras utan att fasen också varieras, och vice versa.

Hur effektiv kommunikationen då är beror mest på mottagningsmetoderna.
I båda fallen uppfattas ändringar i den mottagna signalens frekvens och fasläge.
Amplitudändringar uppfattas däremot inte.
De flesta störningar -- särskilt pulserande sådana som från tändningssystem --
kommer därför att skiljas bort.

För att effektivt utnyttja fördelarna med vinkelmodulation, antingen det är
frekvens eller fasmodulation, behövs tillräckligt frekvensutrymme.
Det innebär att främst högre frekvensband kommer i fråga.
