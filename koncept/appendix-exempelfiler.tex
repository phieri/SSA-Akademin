\chapter{Exempelfiler}
\label{appendix:exempelfiler}
\index{exempelfiler}
\index{GNU Radio}
\index{SPICE}

I den här boken finns exempel- och övningsfiler som kan användas för
att experimentera med radiosystem och elektroniska kretsar.
Filerna finns tillgängliga för nedladdning från projektets GitHub-repository.

\section{Hur man hämtar exempelfilerna}

Den senaste versionen av alla exempelfiler finns i projektets GitHub-repository:

\url{https://github.com/SverigesSandareamatorer/SSA-Akademin/tree/master/exempelfiler}

Du kan ladda ner enskilda filer direkt från GitHub eller klona hela
repositoryt för att få tillgång till alla exempel.

\begin{center}
\qrcode[height=3cm]{https://github.com/SverigesSandareamatorer/SSA-Akademin/tree/master/exempelfiler}
\end{center}

\section{GNU Radio-exempel}
\index{GNU Radio}

GNU Radio är en gratis programvara med öppen källkod för att simulera och
experimentera med radiosystem.
Den kan användas för att visualisera och förstå olika modulationsformer som
beskrivs i kapitel~\ref{ch:modulation}.
GNU Radio finns tillgängligt för Linux, Windows och macOS.
Ladda ner från \url{https://www.gnuradio.org} och följ installationsinstruktionerna.

\subsection{Tillgängliga GNU Radio-exempel}

\textbf{Amplitudmodulation (AM):}
\texttt{am\_modulation.grc}

Detta exempel demonstrerar amplitudmodulation där en lågfrekvent signal
modulerar amplituden på en högfrekvent bärvåg.
Exemplet visar både tidsdomän och frekvensdomän.

\textbf{Frekvensmodulation (FM):}
\texttt{fm\_modulation.grc}

Detta exempel demonstrerar frekvensmodulation där den modulerande signalen
varierar frekvensen på bärvågen.
Exemplet visar både tidsdomän och frekvensdomän.

\section{SPICE-exempel}
\index{SPICE}
\index{LTspice}
\index{ngspice}

SPICE (Simulation Program with Integrated Circuit Emphasis) är en
branschstandard för simulering av elektroniska kretsar.
Det finns flera gratis SPICE-kompatibla program:

\begin{itemize}
\item \textbf{LTspice} -- Gratis från Analog Devices (Windows/macOS)
\item \textbf{ngspice} -- Öppen källkod (Linux/Windows/macOS)
\item \textbf{iCircuit} -- Mobil app för iOS och Android
\end{itemize}

\subsection{Tillgängliga SPICE-exempel}

\textbf{Spänningsdelare:}
\texttt{voltage\_divider.cir}

\mediumfig{images/voltage_divider_circuit.pdf}{Simuleringsresultat för
spänningsdelare -- visar inspänning och utspänning}{fig:voltage-divider-circuit}

En enkel spänningsdelare med två resistorer som demonstrerar Ohms lag och
spänningsfördelning, beskrivet i kapitel~\ref{ch:ellaera}.
Diagrammet visar DC-analysen med inspänning (\qty{12}{\volt}) och utspänning (\qty{6}{\volt}).

\textbf{RC-lågpassfilter:}
\texttt{rc\_lowpass\_filter.cir}

\mediumfig{images/rc_lowpass_circuit.pdf}{Simuleringsresultat för
RC-lågpassfilter -- frekvensrespons och fasförskjutning}{fig:rc-lowpass-circuit}

Ett enkelt första ordningens lågpassfilter som visar hur resistorer och
kondensatorer kan användas för att filtrera signaler.
Diagrammet visar magnitud och fas som funktion av frekvens, med brytfrekvensen
markerad vid cirka \qty{1,59}{\kilo\hertz}.
Se kapitel~\ref{ch:komponenter} och~\ref{ch:kretsar} för mer information.

\textbf{LC-resonanskrets:}
\texttt{lc\_resonant\_circuit.cir}

\mediumfig{images/lc_resonant_circuit.pdf}{Simuleringsresultat för
LC-resonanskrets -- frekvensrespons visar resonanstopp}{fig:lc-resonant-circuit}

En parallell LC-resonanskrets som demonstrerar resonansfenomenet vid en
specifik frekvens, beskrivet i kapitel~\ref{ch:kretsar}.
Diagrammet visar magnitud och fas som funktion av frekvens, med
resonansfrekvensen markerad vid cirka \qty{1,59}{\mega\hertz}.

\section{Experimentera själv}

Dessa exempel är en utgångspunkt för att lära sig mer om radioteknik och
elektronik.
Prova att:

\begin{itemize}
\item Ändra komponentvärden och observera hur resultatet påverkas
\item Kombinera olika kretsar för att skapa mer komplexa system
\item Mäta och jämföra simuleringsresultat med verkliga mätningar
\item Skapa egna exempel baserade på det du lärt dig i boken.
\end{itemize}
