\chapter{Exempelfiler}
\label{appendix:exempelfiler}
\index{exempelfiler}
\index{GNU Radio}
\index{SPICE}

I den här boken finns exempel- och övningsfiler som kan användas för
att experimentera med radiosystem och elektroniska kretsar.
Filerna finns tillgängliga för nedladdning från projektets GitHub-repository.

\section{Hur man hämtar exempelfilerna}

Den senaste versionen av alla exempelfiler finns i projektets GitHub-repository:

\url{https://github.com/SverigesSandareamatorer/SSA-Akademin/tree/master/exempelfiler}

Du kan ladda ner enskilda filer direkt från GitHub eller klona hela
repositoryt för att få tillgång till alla exempel.

\begin{center}
\qrcode[height=3cm]{https://github.com/SverigesSandareamatorer/SSA-Akademin/tree/master/exempelfiler}
\end{center}

\section{GNU Radio-exempel}
\index{GNU Radio}

GNU Radio är en gratis programvara med öppen källkod för simulering av
radiosystem.
Läs mer på \url{https://www.gnuradio.org}.
Exemplen \texttt{am\_modulering.grc} och \texttt{fm\_modulering.grc} demonstrerar
amplitud- och frekvensmodulation.

\section{SPICE-exempel}
\index{SPICE}
\index{LTspice}
\index{ngspice}

SPICE (Simulation Program with Integrated Circuit Emphasis) är en
branschstandard för simulering av elektroniska kretsar.
Det finns flera gratis SPICE-kompatibla program:

\begin{itemize}
\item \textbf{LTspice} -- Gratis från Analog Devices (Windows/macOS)
\item \textbf{ngspice} -- Öppen källkod (Linux/Windows/macOS)
\item \textbf{iCircuit} -- Mobil app för iOS och Android
\end{itemize}

\subsection{Tillgängliga SPICE-exempel}

\textbf{Spänningsdelare:}
\texttt{spanningsdelare.cir}

\mediumfig{images/voltage_divider_circuit.pdf}{Simuleringsresultat för
spänningsdelare -- visar inspänning och utspänning}{fig:voltage-divider-circuit}

En enkel spänningsdelare med två resistorer som demonstrerar Ohms lag och
spänningsfördelning, beskrivet i kapitel~\ref{ch:ellaera}.
Diagrammet visar DC-analysen med inspänning (\qty{12}{\volt}) och utspänning (\qty{6}{\volt}).

\begin{lstlisting}[caption={Källkod för spänningsdelare},label=lst:voltage-divider]
.title Voltage Divider Circuit

* Spänningskälla: 12V DC
Vin input 0 DC 12

* Resistor 1: 10 kOhm
R1 input output 10k

* Resistor 2: 10 kOhm
R2 output 0 10k

* DC operating point analysis
.op

* Skriv ut resultaten
.print dc v(input) v(output) i(Vin)

.control
run
print v(input) v(output) i(Vin)
.endc

.end
\end{lstlisting}

\textbf{RC-lågpassfilter:}
\texttt{rc\_lagpassfilter.cir}

\mediumfig{images/rc_lowpass_circuit.pdf}{Simuleringsresultat för
RC-lågpassfilter -- frekvensrespons och fasförskjutning}{fig:rc-lowpass-circuit}

Ett enkelt första ordningens lågpassfilter som visar hur resistorer och
kondensatorer kan användas för att filtrera signaler.
Diagrammet visar magnitud och fas som funktion av frekvens, med brytfrekvensen
markerad vid cirka \qty{1,59}{\kilo\hertz}.
Se kapitel~\ref{ch:komponenter} och~\ref{ch:kretsar} för mer information.

\begin{lstlisting}[caption={Källkod för RC-lågpassfilter},label=lst:rc-lowpass]
.title RC Low-Pass Filter

* Spänningskälla: 1V AC vid frekvensen 1kHz
Vin input 0 AC 1

* Resistor: 10 kOhm
R1 input output 10k

* Kondensator: 10 nF
C1 output 0 10n

* AC-analys från 10 Hz till 100 kHz med 100 punkter per dekad
.ac dec 100 10 100k

* Skriv ut resultaten
.print ac v(output) vdb(output) vp(output)

.control
run
plot db(v(output))
plot phase(v(output))
.endc

.end
\end{lstlisting}

\section{Experimentera själv}

Dessa exempel är en utgångspunkt för att lära sig mer om radioteknik och
elektronik.
Prova att:

\begin{itemize}
\item Ändra komponentvärden och observera hur resultatet påverkas
\item Kombinera olika kretsar för att skapa mer komplexa system
\item Mäta och jämföra simuleringsresultat med verkliga mätningar
\item Skapa egna exempel baserade på det du lärt dig i boken.
\end{itemize}
