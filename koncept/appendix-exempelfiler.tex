\chapter{Exempelfiler}
\label{app:exempelfiler}
\index{GNU Radio}
\index{SPICE}

Till den här boken finns exempel- och övningsfiler som kan användas för att
experimentera med radiosystem och elektroniska kretsar.

\section{Hur man hämtar exempelfilerna}

Den senaste versionen av alla exempelfiler finns i projektets GitHub-repository:

\url{https://github.com/SverigesSandareamatorer/SSA-Akademin/tree/master/exempelfiler}

Du kan ladda ner enskilda filer direkt från GitHub eller klona hela
repositoryt för att få tillgång till alla exempel.

\begin{center}
\qrcode[height=3cm]{https://github.com/SverigesSandareamatorer/SSA-Akademin/tree/master/exempelfiler}
\end{center}

\section{GNU Radio-exempel}
\index{GNU Radio}

GNU Radio är en gratis programvara med öppen källkod för simulering av
radiosystem.
Läs mer på \url{https://www.gnuradio.org}.
Det finns många bra exempelfiler tillgängliga på GNU Radios officiella
webbplats och i deras dokumentation, inklusive exempel på amplitud- och
frekvensmodulation.

\section{SPICE-exempel}
\index{SPICE}

SPICE (Simulation Program with Integrated Circuit Emphasis) är en
branschstandard för simulering av elektroniska kretsar.
Det finns flera gratis SPICE-kompatibla program:

\begin{itemize}
\item \textbf{ngspice} -- Öppen källkod (Linux/Windows/macOS)
\item \textbf{LTspice} -- Gratis från Analog Devices (Windows/macOS)
\item \textbf{iCircuit} -- Mobilapp för iOS och Android
\end{itemize}

\subsection{Tillgängliga SPICE-exempel}

\textbf{LC-resonanskrets:}
\texttt{resonanskrets.cir}

En parallell LC-resonanskrets som kan användas för att demonstrera
resonans vid en specifik frekvens.

\lstinputlisting[
language=SPICE,
caption={Källkod för LC-resonanskrets},
label=lst:lc-resonanskrets
]{exempelfiler/resonanskrets.cir}

\mediumfig{exempelfiler/resonanskrets.png}{Kretsdiagram för LC-resonanskretsen}{fig:lc-resonanskrets}

\section{Experimentera själv}

Dessa exempel är en utgångspunkt för att lära sig mer om radioteknik och
elektronik.
Prova att

\begin{itemize}
\item ändra komponentvärden och observera hur resultatet påverkas
\item kombinera olika kretsar för att skapa mer komplexa system
\item mäta och jämföra simuleringsresultat med verkliga mätningar
\item skapa egna exempel baserade på det du lärt dig i boken.
\end{itemize}
