\chapter{Exempelfiler}
\label{appendix:exempelfiler}
\index{exempelfiler}
\index{GNU Radio}
\index{SPICE}

I den här boken finns bifogade exempel- och övningsfiler som kan användas för
att experimentera med radiosystem och elektroniska kretsar.
Filerna är inbäddade i PDF:en och kan hämtas ut för användning med olika
simuleringsprogram.

\section{Hur man använder bifogade filer}

De bifogade filerna kan öppnas direkt från PDF-läsaren om den stöder bifogade
filer (t.ex. Adobe Acrobat Reader, Evince eller Okular).
Leta efter symbolen för bifogade filer vid filnamnet.
Du kan också exportera alla bifogade filer från PDF:en med följande kommando:

\texttt{pdftk koncept.pdf unpack\_files output examples/}

\section{GNU Radio-exempel}
\index{GNU Radio}

GNU Radio är en gratis programvara med öppen källkod för att simulera och
experimentera med radiosystem.
Den kan användas för att visualisera och förstå olika modulationsformer som
beskrivs i kapitel~\ref{ch:modulation}.

\subsection{Tillgängliga GNU Radio-exempel}

\textbf{Amplitudmodulation (AM):}
\attachfile[description={GNU Radio exempel för AM-modulation}]{examples/gnuradio/am_modulation.grc}
\texttt{am\_modulation.grc}

Detta exempel demonstrerar amplitudmodulation där en lågfrekvent signal
modulerar amplituden på en högfrekvent bärvåg.
Exemplet visar både tidsdomän och frekvensdomän.

\textbf{Frekvensmodulation (FM):}
\attachfile[description={GNU Radio exempel för FM-modulation}]{examples/gnuradio/fm_modulation.grc}
\texttt{fm\_modulation.grc}

Detta exempel demonstrerar frekvensmodulation där den modulerande signalen
varierar frekvensen på bärvågen.
Exemplet visar både tidsdomän och frekvensdomän.

\subsection{Installation av GNU Radio}

GNU Radio finns tillgängligt för Linux, Windows och macOS.
Ladda ner från \url{https://www.gnuradio.org} och följ installationsinstruktionerna.

\section{SPICE-exempel}
\index{SPICE}
\index{LTspice}
\index{ngspice}

SPICE (Simulation Program with Integrated Circuit Emphasis) är en
branschstandard för simulering av elektroniska kretsar.
Det finns flera gratis SPICE-kompatibla program:

\begin{itemize}
\item \textbf{LTspice} -- Gratis från Analog Devices (Windows/macOS)
\item \textbf{ngspice} -- Öppen källkod (Linux/Windows/macOS)
\item \textbf{iCircuit} -- Mobil app för iOS och Android
\end{itemize}

\subsection{Tillgängliga SPICE-exempel}

\textbf{Spänningsdelare:}
\attachfile[description={SPICE-exempel för spänningsdelare}]{examples/spice/voltage_divider.cir}
\texttt{voltage\_divider.cir}

En enkel spänningsdelare med två resistorer som demonstrerar Ohms lag och
spänningsfördelning, beskrivet i kapitel~\ref{ch:ellaera}.

\textbf{RC-lågpassfilter:}
\attachfile[description={SPICE-exempel för RC-lågpassfilter}]{examples/spice/rc_lowpass_filter.cir}
\texttt{rc\_lowpass\_filter.cir}

Ett enkelt första ordningens lågpassfilter som visar hur resistorer och
kondensatorer kan användas för att filtrera signaler.
Se kapitel~\ref{ch:komponenter} och~\ref{ch:kretsar} för mer information.

\textbf{LC-resonanskrets:}
\attachfile[description={SPICE-exempel för LC-resonanskrets}]{examples/spice/lc_resonant_circuit.cir}
\texttt{lc\_resonant\_circuit.cir}

En parallell LC-resonanskrets som demonstrerar resonansfenomenet vid en
specifik frekvens, beskrivet i kapitel~\ref{ch:kretsar}.

\section{Experimentera själv}

Dessa exempel är en utgångspunkt för att lära sig mer om radioteknik och
elektronik.
Prova att:

\begin{itemize}
\item Ändra komponentvärden och observera hur resultatet påverkas
\item Kombinera olika kretsar för att skapa mer komplexa system
\item Mäta och jämföra simuleringsresultat med verkliga mätningar
\item Skapa egna exempel baserade på det du lärt dig i boken
\end{itemize}

\section{Licens}

Alla exempelfiler är licensierade under Creative Commons
Attribution-ShareAlike 4.0 International (CC BY-SA 4.0), samma licens som
huvudboken.
Du får fritt kopiera, modifiera och dela filerna så länge du anger källan och
delar dina ändringar under samma licens.
