\chapter[Svenska repeatrar]{Frekvenser för amatörradiorepeatrar}
\label{svenska repeatrar}

\noindent Vid direktförbindelser på höga frekvenser är räckvidden begränsad,
särskilt för rörliga (mobila) stationer med låg effekt och små antenner.
Både terräng och teknik begränsar räckvidden.
En repeater med högt belägen antenn ger ofta en bättre möjlighet till förbindelse.
En sådan repeater kan i lyckliga fall inte bara möjliggöra förbindelser, men
ibland också dubblera räckvidden i och med att varje station bara
behöver nå fram till repeatern, se bild \ssaref{fig:bildII7-12}.
En praktisk sak att komma ihåg med repeatrar, speciellt om du sänder med låg
effekt är att du inte alltid når fram till repeatern, även om du kan ta emot
signaler från den.

En repeater sänder omedelbart vad den mottager, så eftersom sändaren och
mottagaren arbetar samtidigt, måste avståndet mellan deras arbetsfrekvenser vara
så stort att det inte uppstår ömsesidiga störningar.
Dessa arbetsfrekvenser kallas \emph{frekvenspar} eller \emph{kanal} och avståndet
mellan dem kallas \emph{repeaterskift}.

Frekvensparet i en repeater måste arbeta med omvänt frekvensläge i förhållande
till det i de stationer som den betjänar.
Kanalavståndet mellan repeatrarna i ett band är också enhetligt och sändningar
över repeatrarna måste naturligtvis ha mindre bandbredd än kanalavståndet.
Inom IARU har man enats bland annat om frekvensparen för smalbandiga
FM-repeatrar.
Se IARU:s bandplaner för \qty{10}{\metre}-repeatrar i bilaga \ssaref{IARU bandplan}.
För VHF-repeatrar finns bandplanen i bilaga \ssaref{Bandplan VHF och högre}.
Frekvensplaner finns för repeatrar inom banden \SIrange{51}{52}{\mega\hertz}
(\qty{6}{\metre}), \SIrange{145}{146}{\mega\hertz} (\qty{2}{\metre}),
\SIrange{432}{438}{\mega\hertz} (\qty{70}{\centi\metre}),
\SIrange{1240}{1300}{\mega\hertz} (\qty{10}{\centi\metre}) samt
\SIrange{28000}{29700}{\kilo\hertz} (\qty{10}{\metre}).

\section{Kanalnumreringsmetod}
% Kan man lägga in några meningar om det gamla sättet att numrera kanaler?
Vid införandet av \qty{12,5}{\kilo\hertz} kanalavstånd på 2~meter- och
\qty{70}{\centi\metre}-banden infördes ett nytt numreringssystem.
Man börjar med en bokstav som talar om vilket band det är och sen en
siffra.

\bigskip
\noindent\begin{tabular}{l|S|S|S}
  \hline
  \multicolumn{1}{l|}{Kod} &
  \multicolumn{1}{l|}{Basfrekvens} &
  \multicolumn{1}{l|}{Kanalavstånd} &
  \multicolumn{1}{l}{Repeatershift}\\
  &
  \multicolumn{1}{l|}{(\unit{\mega\hertz})} &
  \multicolumn{1}{l|}{(\unit{\kilo\hertz})} &
  \multicolumn{1}{l}{(\unit{\kilo\hertz})}\\
      \hline
      H & 29,5 & 10 & -100\\
      F & 51   & 10 & -600\\
      V & 145  & 12,5 & -600\\
      U & 430  & 12,5 & -2000\\
        M & 1240 & 25 & -6000\\
        \hline
\end{tabular}

\bigskip

\noindent Notera att repeatershift kan vara positivt eller negativt, beroende på
var du är i världen.
I vårt närområde är det negativt, men i till exempel Storbritannien så är det
positivt på 70-centimetersbandet. Konsultera alltid lokala bandplaner.

Kanalnumret $n$ börjar med $0$, skrivs 00, på varje sådant band och ökar med ett
(1) för varje kanal i bandet.
För repeaterkanaler sätts ett R före bandbokstaven.

För att räkna fram repeaterns sändningsfrekvens tar man basfrekvensen
$b$ bandet och lägger till kanalavståndet $k$ multiplicerat med
kanalnumret $n$ (\(f = b+kn\)) och för att repeaterns mottagarfrekvens
så lägger man till repeatershift $r$ (\(f = b+kn + r\)).
Repeaterns sändningsfrekvens är din mottagarfrekvens i tabellerna nedan.

Till exampel, låt oss räkna fram frekvenserna för kanal 371 på
70-centimetersbandet.  Då blir basfrekvensen $b = 430\cdot 10^6$,
kanalnumret $n = 371$, kanalavståndet $k = 12,5\cdot 10^3$, och
repeatershift $r = -2000\cdot10^3$.  Notera att basfrekvensen är i
\unit{\mega\hertz} och kanalavstånd samt repeatershift i
\unit{\kilo\hertz}.  Repeaterns sändningsfrekvens blir \(430\cdot 10^6 +
12,5\cdot 10^3 \cdot 371 = 434,6375\cdot10^6 =
\qty{434,6375}{\mega\hertz}\) och dess mottagningsfrekvens blir
\(430\cdot 10^6 + 12,5\cdot 10^3 \cdot 371 + (-2000\cdot 10^3) =
432,6375\cdot10^6 = \qty{432,6375}{\mega\hertz}\).

\section{70-centimetersbandet}

% Repeaterskift negativ \qty{2000}{\kilo\hertz}, med negativ
% repeaterskift så subtraherar repeaterskift från repeaterns
% sändningsfrekvens för att få dess mottagningsfrekvens.
% RU$n$: $430 + 0,0125n$\,\qty{}{\mega\hertz}.

\begin{tabular}{ r | c | l | l }
	Kanal &       & Din sändar-        & Din mottagar-  \\
	Ny    & fd    & frekvens       & frekvens \\
          &       & [\unit{\mega\hertz}] & [\unit{\mega\hertz}] \\
	\hline
	RU361 &       & 432,5125       & 434,5125 DV    \\
	RU362 &       & 432,5250       & 434,5250 DV    \\
	RU363 &       & 432,5375       & 434,5375 DV    \\
	RU364 &       & 432,5500       & 434,5500 DV    \\
	RU365 &       & 432,5625       & 434,5625 DV    \\
	RU366 &       & 432,5750       & 434,5750 DV    \\
	RU367 &       & 432,5875       & 434,5875 DV    \\
	RU368 & RU0   & 432,6000       & 434,6000       \\
	RU369 & RU0x  & 432,6125       & 434,6125       \\
	RU370 & RU1   & 432,6250       & 434,6250       \\
	RU371 & RU1x  & 432,6375       & 434,6375       \\
	RU372 & RU2   & 432,6500       & 434,6500       \\
	RU373 & RU2x  & 432,6625       & 434,6625       \\
	RU374 & RU3   & 432,6750       & 434,6750       \\
	RU375 & RU3x  & 432,6875       & 434,6875       \\
	RU376 & RU4   & 432,7000       & 434,7000       \\
	RU377 & RU4x  & 432,7125       & 434,7125       \\
	RU378 & RU5   & 432,7250       & 434,7250       \\
	RU379 & RU5x  & 432,7375       & 434,7375       \\
	RU380 & RU6   & 432,7500       & 434,7500       \\
	RU381 & RU6x  & 432,7625       & 434,7625       \\
	RU382 & RU7   & 432,7750       & 434,7750       \\
	RU383 & RU7x  & 432,7875       & 434,7875       \\
	RU384 & RU8   & 432,8000       & 434,8000       \\
	RU385 & RU8x  & 432,8125       & 434,8125       \\
	RU386 & RU9   & 432,8250       & 434,8250       \\
	RU387 & RU9x  & 432,8375       & 434,8375       \\
	RU388 & RU10  & 432,8500       & 434,8500       \\
	RU389 & RU10x & 432,8625       & 434,8625       \\
	RU390 & RU11  & 432,8750       & 434,8750       \\
	RU391 & RU11x & 432,8875       & 434,8875       \\
	RU392 & RU12  & 432,9000       & 434,9000       \\
	RU393 & RU12x & 432,9125       & 434,9125       \\
	RU394 & RU13  & 432,9250       & 434,9250       \\
	RU395 & RU13x & 432,9375       & 434,9375       \\
	RU396 & RU14  & 432,9500       & 434,9500       \\
	RU397 & RU14x & 432,9625       & 434,9625       \\
	RU398 & RU15  & 432,9750       & 434,9750       \\
	RU399 & RU15x & 432,9875       & 434,9875       \\
\end{tabular}

\newpage

\section{2-metersbandet}
%% Repeaterskift negativ \qty{600}{\kilo\hertz}, med negativ repeaterskift
%% så subtraherar repeaterskift från repeaterns sändningsfrekvens för att
%% få dess mottagningsfrekvens.
%% RV$n$: $145 + 0,0125n$\,\qty{}{\mega\hertz}.

\begin{tabular}{ r | c | l | l }
	Kanal & & Din sändar- & Din mottagar- \\
	Ny    & fd & frekvens [\unit{\mega\hertz}] & frekvens [\unit{\mega\hertz}] \\
	\hline
	RV46 & & 144,975 & 145,575\\
	RV47 & & 144,9875 & 145,5875\\
	RV48 & R0 & 145,000 & 145,600 \\
	RV49 & R0x & 145,0125 & 145,6125 \\
	RV50 & R1 & 145,025 & 145,625 \\
	RV51 & R1x & 145,0375 & 145,6375 \\
	RV52 & R2 & 145,050 & 145,650 \\
	RV53 & R2x & 145,0625 & 145,6625 \\
	RV54 & R3 & 145,075 & 145,675 \\
	RV55 & R3x & 145,0875 & 145,6875 \\
	RV56 & R4 & 145,100 & 145,700 \\
	RV57 & R4x & 145,1125 & 145,7125 \\
	RV58 & R5 & 145,125 & 145,725 \\
	RV59 & R5x & 145,1375 & 145,7375 \\
	RV60 & R6 & 145,150 & 145,750 \\
	RV61 & R6x & 145,1625 & 145,7625 \\
	RV62 & R7 & 145,175 & 145,775 \\
	RV63 & R7x & 145,1875 & 145,7875 \\
\end{tabular}

\section{23-centimetersbandet}
%% Repeaterskift \qty{6000}{\kilo\hertz}.

\begin{tabular}{ l | l | l }
	Kanal & Din sändar- & Din mottagar- \\
	& frekvens [\unit{\mega\hertz}] & frekvens [\unit{\mega\hertz}] \\
	\hline
	RM0 & 1291,000 & 1297,000 \\
	RM1 & 1291,025 & 1297,025 \\
	RM2 & 1291,050 & 1297,050 \\
	RM3 & 1291,075 & 1297,075 \\
	RM4 & 1291,100 & 1297,100 \\
	RM5 & 1291,125 & 1297,125 \\
	RM6 & 1291,150 & 1297,150 \\
	RM7 & 1291,175 & 1297,175 \\
	RM8 & 1291,200 & 1297,200 \\
	RM9 & 1291,225 & 1297,225 \\
	RM10 & 1291,250 & 1297,250 \\
	RM11 & 1291,275 & 1297,275 \\
	RM12 & 1291,300 & 1297,300 \\
	RM13 & 1291,325 & 1297,325 \\
	RM14 & 1291,350 & 1297,350 \\
	RM15 & 1291,375 & 1297,375 \\
	RM16 & 1291,400 & 1297,450 \\
	RM17 & 1291,425 & 1297,475 \\
	RM18 & 1291,450 & 1297,450 \\
	RM19 & 1291,475 & 1297,475 \\
\end{tabular}

\newpage

\section[Speciella band]{Repeaterband med speciella egenskaper}
\subsection{6-metersbandet}
%% Repeaterskift \qty{600}{\kilo\hertz}.

På grund av den relativt låga frekvensen uppnås ofta överräckvidder på grund av
sporadisk vågutbredning via E-skiktet, se avsnitt \ssaref{e-skiktet}.
Man kan då uppnå förbindelser utan hjälp av repeater.

På \qty{51}{\mega\hertz} så anvånds endast udda kanalnummer.

\bigskip

\begin{tabular}{ l | l | l }
  Kanal & Din sändar- & Din mottagar- \\
        & frekvens [\unit{\mega\hertz}] & frekvens [\unit{\mega\hertz}] \\
  \hline
  RF81 & 51,210 & 51,810 \\
  RF83 & 51,230 & 51,830 \\
  RF85 & 51,250 & 51,850 \\
  RF87 & 51,270 & 51,870 \\
  RF89 & 51,290 & 51,890 \\
  RF91 & 51,310 & 52,910 \\
  RF93 & 51,330 & 52,930 \\
  RF95 & 51,350 & 52,950 \\
  RF97 & 51,370 & 52,970 \\
  RF99 & 51,390 & 52,990 \\
\end{tabular}



\subsection{10-metersbandet}
%% Repeaterskift \qty{100}{\kilo\hertz}.

På grund av den relativt låga frekvensen uppnås stora räckvidder genom
jonosfärisk vågutbredning, särskilt under år med högt solfläckstal.
Även sporadisk vågutbredning via E-skiktet förekommer, se avsnitt
\ssaref{e-skiktet}. I båda fallen bör repeatertrafik undvikas.

\bigskip

\begin{tabular}{ l | l | l }
  Kanal & Din sändar- & Din mottagar- \\
        & frekvens [\unit{\kilo\hertz}] & frekvens [\unit{\kilo\hertz}] \\
  \hline
  RH1 & 29520 & 29620 \\
  RH2 & 29530 & 29630 \\
  RH3 & 29540 & 29640 \\
  RH4 & 29550 & 29650 \\
  RH5 & 29560 & 29660 \\
  RH6 & 29570 & 29670 \\
  RH7 & 29580 & 29680 \\
  RH8 & 29590 & 29690 \\
\end{tabular}
